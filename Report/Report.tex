\documentclass[12pt]{article}
\begin{document}
	\section{Analisi del documento\underline{Piano Triennale per l'informatca nella pubblica amministraione}}
	
	Per prima cosa si parla di come gli enti dovrebbero avere un sistema di gestione informatico strutturato, diciamo ad alto livello, quindi vuol dire che in questo momento, gli enti fanno un uso sregolato della tecnologia, ma nel senso che non ci sono regole comuni di interoperabilità. 
	
	L'Approccio emotivo è quello volto alla centralizzazione delle informazioni, siccome si pensa che mettere tutte le cose importanti in un unico luogo ben gestito possa essere meglio rispetto ad avere un sistema decentralizzato... In particolare, Esempi di quello che si vuole ottenere è: 
	
	•
	il Fascicolo Sanitario Elettronico 2.0;
	•
	la Piattaforma Digitale Nazionale Dati – PDND;
	•
	il Sistema degli Sportelli Unici (SSU) riferito a SUAP/SUE;
	•
	l’implementazione nazionale dello Sportello Digitale Unico europeo - SDG;
	•
	l’ecosistema nazionale di approvvigionamento digitale (e-procurement)
	•
	l’Hub del Turismo Digitale (TDH).
	
	Parola chiave dei servizi - Once only e api first
	
	Il governo dietro a questo progetto di centralizzazione vuole anche fornire ai cittadini l'istruzione necessaria per accedere a questo nuovo paradigma informatico. 
	
	Syllabus per le competenze digitali. 
	
	Usare i fondi del PNRR per il raggiungimento di tale scopo
	
	Quindi fino a questo punto io ho appena appreso che, prima di tutto lo stato vorrebbere creare una rete di comunicazione centralizzata tra i vari enti. questo per definire delle strategia do cuminicazione comuni a tutto gli enti. poi, vorrebbe anche facendo ciò rinforzare la sicurezza dei servizi con il sistema once-only e il API first. Dato questo cambiamento, desiderano creare anche dei corsi formativi per la popolazione e per i dipendenti della pa. tutta questa cosa fatta con i soldi del PNRR 
	
	Cloud first -> dentro i pubblici uffici. Necessità di appaltare servizi esterni, in modo da non avere dentro casa elementi sensibili. -> si vede a questo punto la prima contraddizione, si parla di cloud, quindi di sistema distribuito, quando in realtà la volontà è quella di centralizzare tutto. 
	
	Quindi le informazioni sensibili vanno centralizzate (quindi esternalizzate) mentre gli apparati informatici devono essere esternalizzati a favore di soluzioni cloud -> Questo serve per dimminuire la superfice d'attacco. 
	
	Lo stato italiano è codardo -> questo alla fine è un gioco della deresponsabilizzazione. -> la morale è che la PA non la si vede cone un ente informatico valido, per cui, il lavoro va dato ai professionisti. 
	
	La PA si figura nuovamente come collocamento. 
	
	Dice il documento: 
	
Capitolo 3 - Servizi
Negli ultimi anni, la digitalizzazione è diventata una forza trainante per l'innovazione nei servizi pubblici, con gli enti locali al centro di questo cambiamento.
L'adozione di tecnologie digitali è essenziale per migliorare l'efficienza, aumentare la trasparenza e garantire la qualità dei servizi offerti ai cittadini. In questo processo di trasformazione è indispensabile anche definire un framework di riferimento per guidare ed uniformare le scelte tecnologiche. In particolare, l'architettura a microservizi può esser considerata come una soluzione agile e scalabile, che permette di standardizzare i processi digitali e di facilitare anche il processo di change management nelle organizzazioni governative locali.
Per garantire la possibilità a tutti gli Enti di poter cogliere questa enorme opportunità, anche a coloro che si trovano in condizioni di carenze di know-how e risorse, il presente Piano propone e promuove un’evoluzione del modello di interoperabilità passando dalla sola condivisione dei dati a quella della condivisione dei servizi.
I vantaggi dell’utilizzo di un’architettura basata su micro-servizi sono:
•
Flessibilità e scalabilità
•
Agilità nello sviluppo
•
Integrazione semplificata
•
Resilienza e affidabilità
La transizione verso un'architettura a microservizi richiede la consapevolezza che non sia necessario solo un intervento tecnologico ma che richiede soprattutto un controllo per la gestione del cambiamento che, come abbiamo visto nel cap. 1 coinvolge diverse fasi chiave, quali la formazione continua, il coinvolgimento attivo degli stakeholder, il monitoraggio dell'impatto del cambiamento e naturalmente anche una comunicazione efficace.
Per gli enti locali che potrebbero non avere un know-how interno sufficiente, l'architettura a microservizi offre l'opportunità di sfruttare le soluzioni e i servizi già sviluppati da altri enti. Questo approccio non solo consente di colmare il gap informativo interno, ma fornisce anche un vantaggio significativo in termini di risparmio di tempo e ottimizzazione delle risorse.
L'architettura a microservizi, attraverso la condivisione di processi e lo sviluppo once only riduce la duplicazione degli sforzi e dei costi. La condivisione di e-service vede nella Piattaforma Digitale Nazionale Dati Interoperabilità (PDND) il layer focale per la condivisione di dati e processi.
La sostenibilità e la crescita collaborativa nell'ambito dell'architettura a microservizi non si limita al singolo ente locale. In molte situazioni, possono entrare in gioco altre istituzioni a supporto, come Regioni, Unioni o Enti capofila (HUB tecnologici), che possono agire svolgendo un ruolo fondamentale nello sviluppo fornendo soluzioni tecnologiche e/o amministrative, per facilitare l’integrazione e l’implementazione del processo di innovazione. Questo approccio consente agli enti più piccoli di beneficiare delle risorse condivise e delle soluzioni già implementate, accelerando così il processo di digitalizzazione.
Il coinvolgimento attivo delle istituzioni aggregate come facilitatori tecnologici è essenziale per garantire una transizione armoniosa verso l'architettura a microservizi. Guardando al futuro, la sinergia tra enti locali, Regioni e altre istituzioni aggregate pone le basi per un ecosistema digitale coeso, capace di affrontare sfide complesse e di offrire servizi pubblici sempre più efficienti. La
43
collaborazione istituzionale diventa così un elemento fondamentale per plasmare un futuro digitale condiviso e orientato all'innovazione.
E-Service in interoperabilità tramite PDND
Scenario
L’interoperabilità facilita l'interazione digitale tra Pubbliche Amministrazioni, cittadini e imprese, recependo le indicazioni dell'European Interoperability Framework e, favorendo l’attuazione del principio once only secondo il quale la PA non deve chiedere a cittadini e imprese dati che già possiede.
A fine di raggiungere la completa interoperabilità dei dataset e dei servizi chiave tra le PA centrali e locali e di valorizzare il capitale informativo delle pubbliche amministrazioni, nell’ambito del Sub-Investimento M1C1_1.3.1 “Piattaforma nazionale digitale dei dati” del Piano Nazionale di Ripresa e Resilienza, è stata realizzata la Piattaforma Digitale Nazionale Dati (PDND).
La PDND è lo strumento per gestire l’autenticazione, l’autorizzazione e la raccolta e conservazione delle informazioni relative agli accessi e alle transazioni effettuate suo tramite. La Piattaforma fornisce un insieme di regole condivise per semplificare gli accordi di interoperabilità snellendo i processi di istruttoria, riducendo oneri e procedure amministrative. Un ente può aderire alla Infrastruttura interoperabilità PDND siglando un accordo di adesione, attraverso le funzionalità messe a disposizione dell’infrastruttura.
La PDND permette alle amministrazioni di pubblicare e-service, ovvero servizi digitali conformi alle Linee Guida realizzati ed erogati attraverso l’implementazione di API (Application Programming Interface) REST o SOAP (per retrocompatibilità) cui vengono associati degli attributi minimi necessari alla fruizione. Le API esposte vengono registrate e popolano il Catalogo pubblico degli e-service.
La Piattaforma dovrà evolvere recependo le indicazioni pervenute dalle varie amministrazioni e nel triennio a venire dovrà anche:
1.
consentire la condivisione di dati di grandi dimensioni (bulk) prodotti dalle amministrazioni e l’elaborazione di politiche data-driven;
2.
offrire alle amministrazioni la possibilità di accedere ai dati di enti o imprese di natura privata non amministrativa e di integrarsi con i processi di questi ultimi;
3.
permettere alle amministrazioni di essere informate, in maniera asincrona, su eventuali variazioni a dati precedentemente fruiti, abilitando anche una gestione intelligente dei meccanismi di caching locale delle informazioni;
4.
attivare modelli di erogazione inversa, con i quali un ente, potrà erogare e-service, abilitati a ricevere dati da altri soggetti;
5.
abilitare lo scambio dato sia in modalità sincrona che asincrona, permettendo anche il trasferimento di grosse moli di dati, o di pacchetti dati che necessitano di elevati tempi di elaborazione per il confezionamento;
6.
consentire ad una amministrazione di delegare un altro aderente alla piattaforma ad utilizzare per suo conto le funzionalità dell’infrastruttura medesima per la registrazione, la modifica degli e-service sul Catalogo API e la gestione delle richieste di fruizione degli e-service, ivi compresa la compilazione dell’analisi dei rischi;
7.
pubblicare i propri dati aperti attraverso API che siano catalogate secondo le norme pertinenti (DCAT_AP-IT, INSPIRE, …) e che possano essere raccolte nei portali nazionali ed europei.
44
Al fine di sviluppare servizi integrati sempre più efficienti ed efficaci e di fornire a cittadini e imprese servizi rispondenti alle rispettive esigenze, il Dipartimento per la Trasformazione Digitale supporta le PA nell’adozione del Modello di interoperabilità, pianificando e coordinando iniziative di condivisione, anche attraverso protocolli d'intesa e accordi finalizzati a:
•
costituzione di tavoli e gruppi di lavoro;
•
avvio di progettualità congiunte;
•
capitalizzazione di soluzioni realizzate dalla PA in open source o su siti o forum per condividere la conoscenza (Developers Italia e Forum Italia)


Progettazione dei servizi: accessibilità e design
Scenario
Il miglioramento della qualità e dell’inclusività dei servizi pubblici digitali costituisce la premessa indispensabile per l’incremento del loro utilizzo da parte degli utenti, siano questi cittadini, imprese o altre pubbliche amministrazioni.
Nell’attuale processo di trasformazione digitale è essenziale che i servizi abbiano un chiaro valore per l’utente. Questo obiettivo richiede un approccio multidisciplinare nell'adozione di metodologie e tecniche interoperabili per la progettazione di un servizio. La qualità finale, così come il costo complessivo del servizio, non può infatti prescindere da un’attenta analisi dei molteplici layer, tecnologici e organizzativi interni, che strutturano l’intero processo della prestazione erogata, celandone la complessità sottostante.
Ciò implica anche la necessità di un’adeguata semplificazione dei procedimenti e un approccio sistematico alla gestione dei processi interni, sotto il coordinamento del Responsabile per la transizione al digitale, dotato di un ufficio opportunamente strutturato e con il fondamentale coinvolgimento delle altre strutture responsabili dell’organizzazione e del controllo strategico.
È cruciale, inoltre, il rispetto degli obblighi del CAD in materia di progettazione, accessibilità, privacy, gestione dei dati e riuso, al fine di massimizzare l'efficienza dell'investimento di denaro pubblico e garantire la sovranità digitale con soluzioni software strategiche sotto il completo controllo della Pubblica Amministrazione.
Occorre quindi agire su più livelli e migliorare la capacità delle pubbliche amministrazioni di generare ed erogare servizi di qualità attraverso:
•
l'adozione di modelli e strumenti validati e a disposizione di tutti;
•
il costante monitoraggio da parte delle PA dei propri servizi online;
•
l'incremento del livello di accessibilità dei servizi erogati tramite siti web e app mobile;
•
lo scambio di buone pratiche tra le diverse amministrazioni, da attuarsi attraverso la definizione, la modellazione e l’organizzazione di comunità di pratica;
•
Il riuso e la condivisione di software e competenze tra le diverse amministrazioni.
Per incoraggiare tutti gli utenti a privilegiare il canale online rispetto a quello esclusivamente fisico, rimane necessaria una decisa accelerazione nella semplificazione dell'esperienza d’uso complessiva e un miglioramento dell’inclusività dei servizi, nel pieno rispetto delle norme riguardanti l’accessibilità e il Regolamento generale sulla protezione dei dati.
Per il monitoraggio dei propri servizi, le PA possono utilizzare Web Analytics Italia, una piattaforma
nazionale open source che offre rilevazioni statistiche su indicatori utili al miglioramento continuo
dell’esperienza utente.
Per la realizzazione dei propri servizi digitali, le PA possono utilizzare il Design System del Paese, che consente la realizzazione di interfacce coerenti e accessibili by default, concentrando i budget di progettazione e sviluppo sulle parti e i processi caratterizzanti dello specifico servizio digitale.
Contesto normativo e strategico


Formazione, gestione e conservazione dei documenti informatici
Scenario
Le nuove Linee guida sulla formazione, gestione e conservazione dei documenti informatici dell'Agenzia per l'Italia Digitale, adottate ai sensi dell’art. 71 del CAD e in vigore dal 1° gennaio 2022, rappresentano un importante contributo nel rafforzamento e nell’armonizzazione del quadro normativo di riferimento in tema di produzione, gestione e conservazione dei documenti informatici, mirando a semplificare e rendere più accessibile la materia, integrandola ove necessario, per ricondurla in un unico documento sistematico di pratico utilizzo.
Al loro interno sono delineati i necessari adeguamenti organizzativi e funzionali richiesti alle pubbliche amministrazioni, chiamate a consolidare e rendere concreti i principi di trasformazione digitale enunciati nel CAD e nel Testo Unico sulla Documentazione Amministrativa - TUDA.
Le Linee guida costituiscono la premessa fondamentale dell’agire amministrativo in ambiente digitale, in attuazione degli obiettivi di semplificazione, trasparenza, partecipazione e di economicità, efficacia ed efficienza, già prescritti dalla Legge n.241/1990, assicurando la corretta impostazione metodologica per la loro realizzazione nel complesso percorso di transizione digitale.
La Pubblica Amministrazione è tenuta ad assicurare la rispondenza alle Linee guida, adeguando i propri sistemi di gestione informatica dei documenti, al fine di garantire effetti giuridici conformi alle stesse nei processi documentali, nonché ad ottemperare alle seguenti misure:
•
gestione appropriata dei documenti sin dalla loro fase di formazione per il corretto adempimento degli obblighi di natura amministrativa, giuridica e archivistica tipici della gestione degli archivi pubblici, come delineato nel paragrafo 1.11 delle Linee guida;
•
gestione dei flussi documentali mediante aggregazioni documentali informatiche, come specificato nel paragrafo 3.3;
•
nomina dei ruoli e delle responsabilità previsti, come specificato ai paragrafi 3.1.2 e 4.4;
•
adozione del Manuale di gestione documentale e del Manuale di conservazione, come specificato ai paragrafi 3.5 e 4.7;
•
pubblicazione dei provvedimenti formali di nomina e dei manuali in una parte chiaramente identificabile dell’area “Amministrazione trasparente”, prevista dall’art. 9 del d.lgs. 33/2013;
•
rispetto delle misure minime di sicurezza ICT, emanate da AGID con circolare del 18 aprile 2017, n. 2/2017;
•
rispetto delle in materia di protezione dei dati personali, ai sensi dell’art. 32 del Regolamento UE 679/2016 (GDPR);
•
trasferimento dei documenti al sistema di conservazione, ai sensi del paragrafo 4 e dell’art. 44, comma 1-bis, del CAD.
Il corretto assolvimento di tali obblighi incide significativamente non solo sull'efficacia e l'efficienza della Pubblica Amministrazione, migliorando i processi interni e facilitando gli scambi informativi tra le amministrazioni e il settore privato, ma rappresenta anche un elemento fondamentale nella prestazione di servizi di alta qualità ai cittadini e alle imprese, assicurando trasparenza, accessibilità e protezione di dati e documenti.
Nell'ambito delle sue funzioni di vigilanza, verifica, controllo e monitoraggio, e conformemente a quanto stabilito dall'articolo 18-bis del Codice dell'Amministrazione Digitale - CAD, l'Agenzia per l'Italia Digitale ha pianificato di avviare un'attività di monitoraggio riguardante l'adempimento degli obblighi specificati dalle Linee guida.
52
A questo scopo, entro il 2024 verrà sviluppato un modello basato su indicatori chiari e dettagliati, supportato da un accurato percorso metodologico. Questo permetterà di procedere con un monitoraggio efficace e sistematico, da realizzarsi entro il 2025 per le disposizioni sulla Gestione documentale, e entro il 2026 per quelle relative alla Conservazione digitale.
Contesto normativo


Single Digital Gateway
Scenario
Nel triennio precedente è stata attuata la parte core del Regolamento Europeo EU 2018/1724 sul Single Digital Gateway (SDG) che, con l’obiettivo di costruire uno sportello unico digitale a livello europeo per consentire a cittadini e imprese di esercitare più facilmente i propri diritti e fare impresa all’interno dell’Unione europea, ha di fatto messo online le 21 procedure richieste (19 applicabili in Italia) delle pubbliche amministrazioni direttamente coinvolte in quanto titolari dei servizi.
Il Regolamento, entrato in vigore il 2 ottobre 2018, infatti, ha stabilito le norme per:
1.
l’istituzione e la gestione di uno sportello digitale unico per offrire ai cittadini e alle imprese europee un facile accesso a:
a.
informazioni di alta qualità;
b.
procedure efficienti e interamente online;
c.
servizi di assistenza e di risoluzione dei problemi;
2.
l'uso di procedure da parte di utenti transfrontalieri e l’applicazione del principio once only in accordo con le specifiche normative dei differenti Stati Membri.
A dicembre 2023 AGID ha completato le attività di integrazione e collaudo delle componenti architetturali nazionali SDG, sia per l'interoperabilità tra PA italiane, sia per quella tra PA italiane e quelle degli Stati Membri. Le pubbliche amministrazioni competenti per i procedimenti amministrativi relativi alle procedure (di cui all’Allegato II del Regolamento UE 2018/1724) hanno adeguato i propri procedimenti amministrativi alle specifiche tecniche di implementazione del Single Digital Gateway.
Dopo aver reso disponibile online i servizi relativi delle procedure previste, le attività per il Single Digital Gateway del triennio 2024-2026 riguarderanno prevalentemente azioni di mantenimento,
54
monitoraggio e miglioramento della qualità e dell’accesso ai servizi digitali offerti dallo Sportello per l’Italia, in particolare:
1.
monitoraggio delle componenti nazionali e dei servizi delle PA competenti per garantire l’operatività di tutta la filiera coinvolta nell'attuazione dei processi nazionali e trans-frontalieri del Single Digital Gateway (SDG) attraverso la progettazione e sviluppo di un Operation Center, capace di mettere a sistema tutti gli stakeholder coinvolti che dovranno lavorare in maniera sinergica e standardizzata nella gestione dei processi di operation. Il sistema prevede la predisposizione di un servizio di supporto continuativo di gestione del portafoglio delle applicazioni realizzate e rilasciate, che comprende la presa in carico e la risoluzione delle richieste utente pervenute ad AGID da cittadini e pubbliche amministrazioni (help desk);
2.
supporto alla diffusione dello sportello e del suo utilizzo presso i cittadini e le imprese: rientrano in questa azione attività di supporto alla diffusione dei servizi e attività statistiche di monitoraggio e analisi riguardanti le visite degli utenti alle pagine web impattate dalle singole procedure, nonché al catalogo dei servizi.


Capitolo 4 - Piattaforme
Come per i precedenti Piani, il Piano triennale per l’informatica nella Pubblica Amministrazione 2024-26 prende in esame l’evoluzione delle piattaforme della Pubblica Amministrazione, che offrono funzionalità fondamentali nella digitalizzazione dei processi e dei servizi della PA.
La raggiunta maturità di alcune piattaforme, già presentate nelle precedenti edizioni del Piano, permette qui di focalizzarsi sui servizi che esse offrono a cittadini, a imprese e ad altre amministrazioni, in continuità con quanto descritto nel capitolo precedente “Servizi”.
Nella prima parte di questo capitolo, quindi, si illustrano le piattaforme nazionali che erogano servizi a cittadini e imprese: PagoPA, AppIo, Send, Spid e Cie, NoiPA, Fascicolo sanitario elettronico e SUAP/SUE.
L’obiettivo riferibile a tutte queste piattaforme è comune, si tratta di migliorare i servizi già erogati nei termini che verranno dettagliati nei risultati attesi e nelle linee di azione. In questa sezione, la descrizione di ciascuna piattaforma riporterà lo stesso obiettivo mentre gli altri elementi descrittivi saranno specifici della piattaforma presa in esame. Nella seconda parte di questo capitolo verranno descritte le piattaforme che attestano attributi ed infine si parlerà di basi di dati di interesse nazionale.
Piattaforme nazionali che erogano servizi a cittadini/imprese o ad altre PA
Scenario
pagoPA
pagoPA è la piattaforma che consente ai cittadini di effettuare pagamenti digitali verso la Pubblica Amministrazione in modo veloce e intuitivo. pagoPA offre la possibilità ai cittadini di scegliere tra i diversi metodi di pagamento elettronici in base alle proprie esigenze e abitudini, grazie all’opportunità per i singoli enti pubblici di interfacciarsi con diversi attori del mercato e integrare i propri servizi di incasso con soluzioni innovative. L’obiettivo di pagoPA, infatti, è portare a una maggiore efficienza e semplificazione nella gestione dei pagamenti dei servizi pubblici, sia per i cittadini sia per le amministrazioni, favorendo una costante diminuzione dell’uso del contante.
AppIO
L’app IO è l’esito di un progetto open source nato con l’obiettivo di mettere a disposizione di enti e cittadini un unico canale da cui fruire di tutti i servizi pubblici digitali, quale pilastro della strategia del Governo italiano per la cittadinanza digitale. La visione alla base di IO è mettere al centro il cittadino nell’interazione con la Pubblica Amministrazione, attraverso un’applicazione semplice e intuitiva disponibile direttamente sul proprio smartphone. In particolare, l’app IO rende concreto l’articolo 64 bis del Codice dell’Amministrazione Digitale, che istituisce un unico punto di accesso per tutti i servizi digitali, erogato dalla Presidenza del Consiglio dei Ministri.
SEND
La piattaforma SEND - Servizio Notifiche Digitali (anche noto come Piattaforma Notifiche Digitali di cui all'art. 26 del decreto-legge 76/2020 s.m.i.) rende più veloce, economico e sicuro l’invio e la ricezione delle notifiche a valore legale: permette infatti di riceverle, scaricare i documenti notificati e pagare eventuali spese direttamente online su SEND o nell'app IO.
59
SEND solleva gli enti da tutti gli adempimenti legati alla gestione delle comunicazioni a valore legale e riduce l’incertezza della reperibilità del destinatario.
SPID
L’identità digitale SPID è la soluzione che permette di accedere a tutti i servizi online della Pubblica Amministrazione con un'unica identità digitale. Attraverso credenziali classificate su tre livelli di sicurezza, abilita ad accedere ai servizi, ai quali fornisce dati identificativi certificati.
SPID è costituito come insieme aperto di soggetti pubblici e privati che, previo accreditamento da parte dell'Agenzia per l'Italia Digitale, gestiscono i servizi di registrazione e di messa a disposizione delle credenziali e degli strumenti di accesso in rete nei riguardi di cittadini e imprese.
A dicembre 2023 sono state rilasciate ai cittadini oltre 36 milioni e mezzo di identità digitali SPID, che hanno permesso nel 2023 di totalizzare oltre 1.000.000.000 di autenticazioni a servizi online di pubbliche amministrazioni e privati. Attualmente la federazione SPID è composta da più di 15.000 fornitori di servizi pubblici e 177 fornitori di servizi privati.
Nell’ambito del PNRR il sub-investimento M1C1 1.4.4 “Rafforzamento dell'adozione delle piattaforme nazionali di identità digitale (SPID, CIE) e dell'Anagrafe nazionale della popolazione residente (ANPR)”, di cui è soggetto titolare il Dipartimento per la Trasformazione Digitale della Presidenza del Consiglio dei Ministri, include fra le sue finalità che i gestori delle identità SPID assicurino l'innalzamento del livello dei servizi, della qualità, sicurezza e di interoperabilità degli stessi stabiliti dalle Linee guida AGID, come previsto dall’art. 18 bis del D.L. 24/02/2023 n. 13, convertito dalla L. 21/04/2023 n. 41.
A tal fine, è necessario che il Sistema SPID evolva in base alle seguenti indicazioni:
•
attuazione delle “Linee guida OpenID Connect in SPID” (Determinazione del Direttore Generale di AGID n. 616/2021) comprensive dell’Avviso SPID n. 41 del 23/3/2023 versione 2.0 e il “Regolamento - SPID OpenID Connect Federation 1.0” (Determinazione del Direttore Generale di AGID n. 249/2022);
•
attuazione delle “Linee guida operative per la fruizione dei servizi SPID da parte dei minori” (Determinazione del Direttore Generale di AGID n. 133/2022);
•
attuazione delle “Linee guida recanti le regole tecniche dei Gestori di attributi qualificati” (Determinazione del Direttore Generale di AGID n. 215/2022);
•
promozione dell’utilizzo dello SPID dedicato all’uso professionale per l’accesso ai servizi online rivolti a professionisti e imprese.
CIE
L’identità digitale CIE (CIEId), sviluppata e gestita dall’Istituto Poligrafico e Zecca dello Stato, consente la rappresentazione informatica della corrispondenza tra un utente e i suoi attributi identificativi, ai sensi del CAD, verificata attraverso l’insieme dei dati raccolti e registrati in forma digitale al momento del rilascio della CIE. La CIEId è comprovata dal cittadino attraverso l’uso della CIE o delle credenziali rilasciate dal Ministero.
Alla data di metà dicembre 2023 sono state rilasciate ai cittadini oltre 40 milioni di Carte di Identità Elettroniche, che hanno permesso nel 2023 di totalizzare circa 32.000.000 di autenticazioni a servizi online di pubbliche amministrazioni e privati. Attualmente la federazione CIE è composta da più di 10.000 fornitori di servizi pubblici e circa 100 fornitori di servizi privati.
Come sancito dal Decreto 8 settembre 2022 “Modalità di impiego della carta di identità elettronica”, sono previste le seguenti evolutive sul servizio CIEId:
60
1.
Ampliamento del set di attributi forniti tramite autenticazione con CIEId, come previsto dall’art. 6;
2.
ampliamento delle funzionalità del portale del cittadino, come previsto dall’art. 14, tra cui la possibilità di visualizzare, esprimere o revocare la volontà in merito alla donazione di organi e tessuti;
3.
implementazione dei servizi correlati al NIS (Numero Identificativo Servizi), come previsto dall’art. 17;
4.
implementazione di una piattaforma di firma elettronica qualificata remota attraverso l’utilizzo della CIE;
5.
implementazione dell’integrazione con il sistema ANPR, al fine di ricevere giornalmente i dati afferenti ai soggetti deceduti e procedere al blocco tempestivo della CIEId;
6.
sviluppo di un meccanismo di controllo genitoriale per consentire un accesso controllato ai servizi online offerti ai minori.
NoiPA
NoiPA è la piattaforma dedicata a tutto il personale della Pubblica Amministrazione, che offre servizi evoluti per la gestione, integrata e flessibile, di tutti i processi in ambito HR, inclusi i relativi adempimenti previsti dalla normativa vigente. Inoltre, attraverso il portale Open Data NoiPA, è possibile la piena fruizione dell’ampio patrimonio informativo gestito, permettendo la consultazione, in forma aggregata, dei dati derivanti dalla gestione del personale delle pubbliche amministrazioni servite.
Fascicolo Sanitario Elettronico
Il Fascicolo Sanitario Elettronico (FSE 2.0) ha l‘obiettivo di garantire la diffusione e l’accessibilità dei servizi di sanità digitale in modo omogeneo e capillare su tutto il territorio nazionale a favore dei cittadini e degli operatori sanitari delle strutture pubbliche, private accreditate e private.
La verifica formale e semantica della corretta implementazione e strutturazione dei documenti secondo gli standard ha lo scopo di assicurare omogeneità a livello nazionale per i servizi del FSE 2.0 disponibili ai cittadini e ai professionisti della Sanità.
Attraverso interventi sistematici di formazione, si intende superare le criticità legate alle competenze digitali dei professionisti del sistema sanitario, innalzandone significativamente il livello per un utilizzo pieno ed efficace del FSE 2.0.
SUAP e SUE
Nel panorama della Pubblica Amministrazione, gli Sportelli Unici per le Attività Produttive (SUAP) e per l'Edilizia (SUE) assumono un ruolo centrale come punto di convergenza per imprese, professionisti e cittadini nell'interazione con le istituzioni, nell’ambito degli adempimenti previsti per le attività produttive (quali, ad esempio, la produzione di beni e servizi, le attività agricole, commerciali e artigianali, le attività turistiche alberghiere ed extra-alberghiere, i servizi resi dalle banche e dagli intermediari finanziari e i servizi di telecomunicazione, ecc.) e gli interventi edilizi. Si tratta di due pilastri fondamentali in un contesto in continua evoluzione, dove la digitalizzazione si configura non solo come una necessità imprescindibile, ma anche come una leva strategica fondamentale per favorire la competitività delle imprese, stimolare la crescita economica del Paese e ottimizzare la tempestività nell'evasione delle richieste. In questo scenario, la semplificazione e l'accelerazione dei procedimenti amministrativi diventano così il mezzo con cui costruire un futuro in cui le opportunità digitali diventino accessibili a tutti.
61
Nell’ambito delle iniziative previste dal Piano Nazionale di Ripresa e Resilienza (PNRR), è stato avviato il percorso di trasformazione incentrato sulla digitalizzazione e la semplificazione dei sistemi informatici, partendo dalla redazione delle Specifiche tecniche, elaborate attraverso il lavoro congiunto del Gruppo tecnico (istituito dal Ministero delle Imprese e del Made in Italy e dal Dipartimento della Funzione Pubblica e coordinato dall’Agenzia per l’Italia Digitale), le quali delineano l’insieme delle regole e delle modalità tecnologiche che i Sistemi Informatici degli Sportelli Unici (SSU) devono adottare, per la gestione ottimale dei procedimenti amministrativi riguardanti le attività produttive, conformemente alle disposizioni del DPR 160/2010 e ss.mm.ii.
La fase operativa di questo percorso è stata condotta partendo da un’attenta analisi della situazione esistente, rafforzata, successivamente, dalla somministrazione di un questionario di valutazione, volto ad identificare la maturità tecnologica iniziale degli sportelli unici, grazie alla diretta collaborazione delle amministrazioni coinvolte. Attualmente, è terminata la raccolta delle informazioni, perfezionata con altre attività di indagine, come la consulta dei fornitori dei servizi IT relativi alle piattaforme, i tavoli tematici regionali e il coinvolgimento di altri stakeholder e si sta procedendo con la definizione dei piani di intervento, da realizzarsi attraverso risorse finanziarie messe a disposizione dal Dipartimento della Funzione Pubblica, tramite la pubblicazione di bandi/stipula di accordi per l’adeguamento delle piattaforme.
In tale percorso di trasformazione, che vedrà impegnate le pubbliche amministrazioni nel prossimo triennio, per garantire il raggiungimento delle milestone definite nell’ambito del PNRR, deve essere assicurato il supporto tecnico necessario all’adeguamento delle soluzioni informatiche alle Specifiche tecniche, attraverso la condivisione delle conoscenze e dell’esperienza maturata nel campo, utili a fornire una corretta interpretazione delle stesse durante la fase di realizzazione degli interventi.
Contesto normativo e strategico
In materia di Piattaforme esistono una serie di riferimenti, normativi o di indirizzo, cui le Amministrazioni devono attenersi. Di seguito si riporta un elenco delle principali fonti, generali o specifiche, della singola piattaforma citata nel capitolo:


Piattaforme che attestano attributi
Scenario
Negli ultimi anni le iniziative intraprese dai vari attori coinvolti nell’ambito del Piano, hanno favorito una importante accelerazione nella diffusione di alcune delle principali piattaforme abilitanti, in termini di adozione da parte delle PA e di fruizione da parte degli utenti. Il Piano descrive lo sviluppo di nuove piattaforme e il consolidamento di quelle già in essere attraverso l’aggiunta di nuove funzionalità. Tali piattaforme rendono disponibili i dati di settore ai cittadini e PA, consentono di razionalizzare i servizi per le amministrazioni e di semplificare tramite l’utilizzo delle tecnologie digitali l’interazione tra cittadini e PA (per la Piattaforma Digitale Nazionale Dati – PDND).
Ad esempio, nel luglio 2023 la Piattaforma INAD è andata in esercizio, in consultazione, sia tramite il sito web sia tramite le API esposte su PDND, attualmente in esercizio. La piattaforma è quindi a disposizione per entrambe le modalità di fruizione, da parte delle pubbliche amministrazioni. Si invitano pertanto le PA a fruire dei relativi servizi, compatibilmente con il loro dimensionamento.
In questo ambito vengono attuate le seguenti Piattaforme che hanno la caratteristica di attestare attributi anagrafici e di settore.
ANPR: è l’Anagrafe Nazionale che raccoglie tutti i dati anagrafici dei cittadini residenti in Italia e dei cittadini italiani residenti all’estero, aggiornata con continuità dagli oltre 7900 comuni italiani, consentendo di avere un set di dati anagrafici dei cittadini certo, accessibile, affidabile e sicuro su cui sviluppare servizi integrati ed evoluti per semplificare e velocizzare le procedure tra Pubbliche amministrazioni e con il cittadino.
Sul portale ANPR, nell’area riservata del cittadino, sono attualmente disponibili i servizi che consentono al cittadino di:
•
visualizzare i propri dati anagrafici;
•
effettuare una richiesta di rettifica per errori materiali;
•
richiedere autocertificazioni precompilate con i dati anagrafici presenti in ANPR;
•
richiedere un certificato anagrafico in bollo o in esenzione (sono disponibili 15 tipologie differenti di certificati);
•
comunicare un cambio di residenza;
•
visualizzare il proprio domicilio digitale, costantemente allineato con l’Indice Nazionale dei Domicili Digitali (INAD);
•
comunicare un punto di contatto (mail o telefono).
A dicembre 2022 sono stati resi disponibili i servizi per consentire, da parte dei Comuni, l'invio dei dati elettorali dei cittadini in ANPR. Attualmente oltre il 97% dei comuni italiani hanno aderito ai servizi, inviando i dati elettorali dei cittadini.
70
La presenza dei dati elettorali in ANPR consentirà ai cittadini di visualizzare nell’area riservata i dati relativi alla propria posizione elettorale e richiedere certificati di godimento dei diritti politici e di iscrizione nelle liste elettorali.
Inoltre, consentirà di verificare in tempo reale la posizione elettorale di un cittadino da parte di altre Amministrazioni che ne abbiano necessità per fini istituzionali. Una prima applicazione si avrà con l’integrazione dei servizi ANPR con la Piattaforma Referendum, piattaforma online che consentirà la sottoscrizione di proposte referendarie e di iniziativa popolare, verificando in tempo reale la posizione elettorale del cittadino sottoscrittore.
Al fine di agevolare lo sviluppo di sistemi integrati ed evoluti, che semplifichino e velocizzino le procedure tra le Pubbliche Amministrazioni, ANPR ha reso disponibili 28 e-service sulla Piattaforma Nazionale Digitale Dati (PDND) - Interoperabilità, consentendo la consultazione dei dati ANPR da parte di altri Enti aventi diritto, nel rispetto dei principi del Regolamento Privacy.
In aggiunta, l’integrazione dell’ANPR con i servizi dello Stato civile digitale ha un rilievo centrale e strategico nel processo di digitalizzazione della Pubblica Amministrazione e costituisce un significativo strumento di semplificazione per i Comuni e per i cittadini. Si prevede, infatti, la completa digitalizzazione dei registri dello Stato civile tenuti dai Comuni (nascita, matrimonio, unione civile, cittadinanza e morte), con conseguente eliminazione dei registri cartacei, e la conservazione dei relativi atti digitali in un unico archivio nazionale del Ministero dell’Interno, permettendone la consultazione a livello nazionale e offrendo la possibilità di produrre estratti o certificati tramite il sistema centrale, senza doverli richiedere necessariamente al Comune che li ha generati. Alcuni Comuni pilota ad ottobre 2023 hanno iniziato ad utilizzare i servizi resi disponibili da ANPR, formando atti digitali di stato civile con effetti giuridici.
ANPR si sta integrando con le anagrafi settoriali del lavoro, della pensione e del welfare e ogni nuova anagrafe che abbia come riferimento la popolazione residente sarà logicamente integrata con ANPR.
In questo contesto, per rafforzare gli interventi nei settori di istruzione, università e ricerca, accelerare il processo di automazione amministrativa e migliorare i servizi per i cittadini e le pubbliche amministrazioni, sono istituite due Anagrafi:
•
ANIST: l'Anagrafe nazionale dell'istruzione, a cura del Ministero dell’Istruzione e del Merito
•
ANIS: l'Anagrafe nazionale dell'istruzione superiore, a cura del Ministero dell’Università e della Ricerca.
Le due Anagrafi mirano ad assicurare:
•
La centralizzazione dei dati attualmente distribuiti su tutto il territorio italiano in oltre 10.000 scuole (ANIST) e 500 istituti di formazione superiore (ANIS);
•
la disponibilità e l’accesso ai dati per:
o
scuole e istituti di formazione superiore (IFS), al fine di facilitare il reperimento delle informazioni relative al percorso scolastico e/o accademico dei propri studenti, efficientando le procedure di iscrizione;
o
cittadini, al fine rendere possibile, attraverso il Portale dedicato, la consultazione online dei dati relativi al proprio percorso scolastico e/o accademico, anche a fini certificativi;
o
PA per fini istituzionali;
o
soggetti privati autorizzati, per gli scopi previsti dalla legge.
•
l'interoperabilità con altre banche dati (es. con ANPR per la gestione dei dati anagrafici degli studenti, eliminando duplicazioni e rischi di disallineamento);
•
il riconoscimento nell’UE e extra-EU dei titoli di studio.
Per l’avvio progettuale di ANIST si attende la conclusione del relativo iter normativo.

Intelligenza artificiale per la Pubblica Amministrazione
Scenario
Per sistema di Intelligenza Artificiale (IA) si intende un sistema automatico che, per obiettivi espliciti o impliciti, deduce dagli input ricevuti come generare output come previsioni, contenuti, raccomandazioni o decisioni che possono influenzare ambienti fisici o virtuali. I sistemi di IA variano nei loro livelli di autonomia e adattabilità dopo l'implementazione (Fonte: OECD AI principles overview).
Figura 3 - Sistema di intelligenza artificiale (Fonte OECD)
L'intelligenza artificiale ha il potenziale per essere una tecnologia estremamente utile, o addirittura dirompente, per la modernizzazione del settore pubblico. L'IA sembra essere la risposta alla crescente necessità di migliorare l'efficienza e l'efficacia nella gestione e nell'erogazione dei servizi pubblici. Tra le potenzialità delle tecnologie di intelligenza artificiale si possono citare le capacità di:
•
automatizzare attività di ricerca e analisi delle informazioni semplici e ripetitive, liberando tempo di lavoro per attività a maggior valore;
•
aumentare le capacità predittive, migliorando il processo decisionale basato sui dati;
•
supportare la personalizzazione dei servizi incentrata sull'utente, aumentando l'efficacia dell'erogazione dei servizi pubblici anche attraverso meccanismi di proattività.
L'Unione Europea mira a diventare leader strategico nell'impiego dell'intelligenza artificiale nel settore pubblico. Questa intenzione è chiaramente espressa nella Comunicazione “Piano Coordinato sull'Intelligenza Artificiale” COM (2021) 205 del 21 aprile 2021 in cui la Commissione europea propone specificamente di "rendere il settore pubblico un pioniere nell'uso dell'IA".
La revisione del Piano sull’intelligenza artificiale è stata accompagnata dalla “Proposta di Regolamento del Parlamento Europeo e del Consiglio che stabilisce regole armonizzate sull’intelligenza artificiale” (AI Act) COM (2021) 206 del 21 aprile 2021. La proposta di regolamento mira ad affrontare i rischi legati all’utilizzo dell'IA, classificandoli in quattro diversi livelli: rischio inaccettabile (divieto), rischio elevato, rischio limitato e rischio minimo. Inoltre, il regolamento intende porre le basi per costruire un ecosistema di eccellenza nell'IA e rafforzare la capacità dell'Unione Europea di competere a livello globale.
L’AI Act ha introdotto una importante sfida in materia di normazione tecnica. La Commissione Europea ha adottato il 25 maggio 2023 la Decisione C(2023)3215 - Standardisation request M/5932 con la quale ha affidato agli Enti di normazione europei CEN e CENELEC la redazione di norme tecniche europee a vantaggio dei sistemi di intelligenza artificiale in conformità con i principi dell’AI Act.
87
Il “Dispositivo per la ripresa e la resilienza” ha tra gli obiettivi quello di favorire la creazione di una industria dell’intelligenza artificiale nell’Unione Europea al fine di assumere un ruolo guida a livello globale nello sviluppo e nell'adozione di tecnologie di IA antropocentriche, affidabili, sicure e sostenibili. In Italia il PNRR prevede importanti misure di finanziamento sia per la ricerca in ambito di intelligenza artificiale sia per lo sviluppo di piattaforme di IA per i servizi della Pubblica Amministrazione.
Il DTD di concerto con ACN e AGID promuoverà l’obiettivo di innalzare i livelli di cybersecurity dell’Intelligenza Artificiale per assicurare che sia progettata, sviluppata e impiegata in maniera sicura, anche in coerenza con le linee guida internazionali sulla sicurezza dell’Intelligenza Artificiale. La cybersecurity è un requisito essenziale dell’IA e serve per garantire resilienza, privacy, correttezza ed affidabilità, ovvero un cyberspazio più sicuro.
La Pubblica Amministrazione italiana conta esperienze rilevanti nello sviluppo e utilizzo di soluzioni di intelligenza artificiale. A titolo esemplificativo si citano le esperienze di:
•
Agenzia delle entrate, utilizzo di algoritmi di machine learning per analizzare schemi e comportamenti sospetti, aiutando nella prevenzione e rilevazione di frodi;
•
INPS, adozione di chatbot per semplificare e personalizzare l'interazione con l’utente, migliorando l'accessibilità e l'usabilità dei servizi;
•
ISTAT, utilizzo di foundation models per generare ontologie a partire dalla descrizione in linguaggio naturale del contesto semantico al fine di migliorare la qualità della modellazione dei dati.
In questo contesto, l’affermarsi dei foundation models costituisce un importante fattore di accelerazione per lo sviluppo e l’adozione di soluzioni di intelligenza artificiale. Per foundation models si intendono sistemi di grandi dimensioni in grado di svolgere un'ampia gamma di compiti specifici, come la generazione di video, testi, immagini, la conversazione in linguaggio naturale, l'elaborazione o la generazione di codice informatico. L’AI Act definisce inoltre come foundation models “ad alto impatto” i modelli addestrati con una grande quantità di dati e con complessità, capacità e prestazioni elevate.
Principi generali per l’utilizzo dell’intelligenza artificiale nella Pubblica Amministrazione
Le amministrazioni pubbliche devono affrontare molte sfide nel perseguire l'utilizzo dell'intelligenza artificiale. Di seguito si riportano alcuni principi generali che dovranno essere adottati dalle pubbliche amministrazioni e declinati in fase di applicazione tenendo in considerazione lo scenario in veloce evoluzione.
1.
Miglioramento dei servizi e riduzione dei costi. Le pubbliche amministrazioni concentrano l’investimento in tecnologie di intelligenza artificiale nell’automazione dei compiti ripetitivi connessi ai servizi istituzionali obbligatori e al funzionamento dell'apparato amministrativo. Il conseguente recupero di risorse è destinato al miglioramento della qualità dei servizi anche mediante meccanismi di proattività.
2.
Analisi del rischio. Le amministrazioni pubbliche analizzano i rischi associati all'impiego di sistemi di intelligenza artificiale per assicurare che tali sistemi non provochino violazioni dei diritti fondamentali della persona o altri danni rilevanti. Le pubbliche amministrazioni adottano la classificazione dei sistemi di IA secondo le categorie di rischio definite dall’AI Act.
3.
Trasparenza, responsabilità e informazione. Le pubbliche amministrazioni pongono particolare attenzione alla trasparenza e alla interpretabilità dei modelli di intelligenza artificiale al fine di garantire la responsabilità e rendere conto delle decisioni adottate con il
88
supporto di tecnologie di intelligenza artificiale. Le amministrazioni pubbliche forniscono
informazioni adeguate agli utenti al fine di consentire loro di prendere decisioni informate riguardo all'utilizzo dei servizi che sfruttano l'intelligenza artificiale.
4.
Inclusività e accessibilità. Le pubbliche amministrazioni sono consapevoli delle responsabilità e delle implicazioni etiche associate all'uso delle tecnologie di intelligenza artificiale. Le pubbliche amministrazioni assicurano che le tecnologie utilizzate rispettino i principi di equità, trasparenza e non discriminazione.
5.
Privacy e sicurezza. Le pubbliche amministrazioni adottano elevati standard di sicurezza e protezione della privacy per garantire che i dati dei cittadini siano gestiti in modo sicuro e responsabile. In particolare, le amministrazioni garantiscono la conformità dei propri sistemi di IA con la normativa vigente in materia di protezione dei dati personali e di sicurezza cibernetica.
6.
Formazione e sviluppo delle competenze. Le pubbliche amministrazioni investono nella formazione e nello sviluppo delle competenze necessarie per gestire e applicare l’intelligenza artificiale in modo efficace nell’ambito dei servizi pubblici. A tale proposito si faccia riferimento agli obiettivi individuati nel Capitolo 1.
7.
Standardizzazione. Le pubbliche amministrazioni tengono in considerazione, durante le fasi di sviluppo o acquisizione di soluzioni basate sull'intelligenza artificiale, le attività di normazione tecnica in corso a livello internazionale e a livello europeo da CEN e CENELEC con particolare riferimento ai requisiti definiti dall’AI Act.
8.
Sostenibilità: Le pubbliche amministrazioni valutano attentamente gli impatti ambientali ed energetici legati all’adozione di tecnologie di intelligenza artificiale e adottando soluzioni sostenibili dal punto di vista ambientale.
9.
Foundation Models (Sistemi IA “ad alto impatto”). Le pubbliche amministrazioni, prima di adottare foundation models “ad alto impatto”, si assicurano che essi adottino adeguate misure di trasparenza che chiariscono l’attribuzione delle responsabilità e dei ruoli, in particolare dei fornitori e degli utenti del sistema di IA.
10.
Dati. Le pubbliche amministrazioni, che acquistano servizi di intelligenza artificiale tramite API, valutano con attenzione le modalità e le condizioni con le quali il fornitore del servizio gestisce di dati forniti dall’amministrazione con particolare riferimento alla proprietà dei dati e alla conformità con la normativa vigente in materia di protezione dei dati e privacy.
Dati per l’intelligenza artificiale
La disponibilità di dati di alta qualità e il rispetto dei valori e dei diritti europei, quali la protezione dei dati personali, la tutela dei consumatori e la normativa in materia di concorrenza sono i prerequisiti fondamentali nonché un presupposto per lo sviluppo e la diffusione dei sistemi di IA. La disponibilità di dati rappresenta peraltro un requisito chiave per l’adozione di un approccio all’intelligenza artificiale attento alle specificità nazionali.
La Strategia Europea per i dati è implementata dal punto normativo dagli atti sopra citati che costituiscono il quadro regolatorio entro il quale deve muoversi una Pubblica Amministrazione che intende operare con sistemi di IA sui dati aperti.
Riguardo l’utilizzo dei dati da parte di sistemi di intelligenza artificiale, l’AI Act richiede ai fornitori di sistemi di IA di adottare una governance dei dati e appropriate procedure di gestione dei dati (con particolare attenzione alla generazione e alla raccolta dei dati, alle operazioni di preparazione dei dati, alle scelte di progettazione e alle procedure per individuare e affrontare le distorsioni e le potenziali distorsioni per correlazione o qualsiasi altra carenza pertinente nei dati). L’AI Act pone particolare attenzione agli aspetti qualitativi dei set di dati utilizzati per addestrare, convalidare e testare i sistemi
89
di IA (tra cui rappresentatività, pertinenza, completezza e correttezza). La Commissione Europea ha avviato una specifica attività presso il CEN e il CENELEC per definire norme tecniche europee per rispondere a tali esigenze.
Nel contesto nazionale, tenuto conto di una architettura istituzionale che organizza i territori in regioni e comuni, che devono avere livelli di servizio omogenei, diventa cruciale progettare e implementare soluzioni nazionali basate sull'IA. Queste soluzioni devono essere in grado, da un lato, di superare eventuali disparità che caratterizzano le diverse amministrazioni territoriali e, dall'altro, di assicurare un pieno coordinamento tra territori differenti riguardo a servizi chiave per la società.
Riguardo l’affermarsi dei foundation models nel settore pubblico, una sfida fondamentale consiste nella creazione di dataset di elevata qualità, rappresentativi della realtà della Pubblica Amministrazione, con particolare riguardo al corpus normativo nazionale e comunitario, ai procedimenti amministrativi e alla struttura organizzativa della Pubblica Amministrazione italiana stessa.
Contesto normativo e strategico
Riferimenti normativi europei:
•
Comunicazione della Commissione al Parlamento Europeo e al Consiglio, “Piano Coordinato sull'Intelligenza Artificiale”, COM (2021) 205 del 21 aprile 2021
•
“Proposta di Regolamento del Parlamento Europeo e del Consiglio che stabilisce regole armonizzate sull’intelligenza artificiale” (AI Act), COM (2021) 206, del 21 aprile 2021
•
Decisione della Commissione “on a standardisation request to the European Committee for Standardisation and the European Committee for Electrotechnical Standardisation in support of Union policy on artificial intelligence” C (2023) 3215 del 22 maggio 2023

\underline{L'argomento estratto che parla della intelligenza artificiale è solo una curiosità ai fini del report che si vuol sviluppare}
	
Capitolo 6 - Infrastrutture
Infrastrutture digitali e Cloud
Scenario
La strategia “Cloud Italia”, pubblicata a settembre 2021 dal Dipartimento per la Trasformazione Digitale e dall’Agenzia per la Cybersicurezza Nazionale nell’ambito del percorso attuativo definito dall’art.33-septies del Decreto-Legge n.179 del 2012 e gli investimenti del PNRR legati all’abilitazione cloud rappresentano una grande occasione per supportare la riorganizzazione strutturale e gestionale delle pubbliche amministrazioni.
Non si tratta di una operazione unicamente tecnologica, le cui opportunità vanno esplorate a fondo da ogni ente.
La Strategia Cloud risponde a tre sfide principali: assicurare l’autonomia tecnologica del Paese, garantire il controllo sui dati e aumentare la resilienza dei servizi digitali. In coerenza con gli obiettivi del PNRR, la strategia traccia un percorso per accompagnare le PA italiane nella migrazione dei dati e degli applicativi informatici verso un ambiente cloud sicuro.
Con il principio cloud first, si vuole guidare e favorire l’adozione sicura, controllata e completa delle tecnologie cloud da parte del settore pubblico, in linea con i principi di tutela della privacy e con le raccomandazioni delle istituzioni europee e nazionali. In particolare, le pubbliche amministrazioni, in fase di definizione di un nuovo progetto, e/o di sviluppo di nuovi servizi, in via prioritaria devono valutare l’adozione del paradigma cloud prima di qualsiasi altra tecnologia.
Secondo tale principio, quindi, tutte le Amministrazioni sono obbligate ad effettuare una valutazione in merito all’adozione del cloud che rappresenta l’evoluzione tecnologica più dirompente degli ultimi anni e che sta trasformando radicalmente tutti i sistemi informativi della società a livello mondiale. Nel caso di eventuale esito negativo, tale valutazione dovrà essere motivata.
L’adozione del paradigma cloud rappresenta, infatti, la chiave della trasformazione digitale abilitando una vera e propria rivoluzione del modo di pensare i processi di erogazione dei servizi della PA verso cittadini, professionisti ed imprese.
L’attuazione dell’art.33-septies del Decreto-legge n. 179 del 2012, non rappresenta solo un adempimento legislativo, ma è soprattutto una occasione perché ogni ente attivi gli opportuni processi di gestione interna con il fine di modernizzare i propri applicativi e al contempo migliorare la fruizione dei procedimenti, delle procedure e dei servizi erogati.
È anche quindi una grande occasione per:
•
ridurre il debito tecnologico accumulato negli anni dalle amministrazioni;
•
mitigare il rischio di lock-in verso i fornitori di sviluppo e manutenzione applicativa;
•
ridurre significativamente i costi di manutenzione di centri elaborazione dati (data center) obsoleti e delle applicazioni legacy, valorizzando al contempo le infrastrutture digitali del Paese più all’avanguardia che stanno attuando il percorso di adeguamento rispetto ai requisiti del Regolamento AGID e relativi atti successivi dell’Agenzia per la Cybersicurezza Nazionale;
•
Incrementare la postura di sicurezza delle infrastrutture pubbliche per proteggerci dai rischi cyber.
94
In tal modo, le infrastrutture digitali saranno più affidabili e sicure e la Pubblica Amministrazione potrà rispondere in maniera organizzata agli attacchi informatici, garantendo continuità e qualità nella fruizione di dati e servizi.
Nell’ambito dell’attuazione normativa della Strategia Cloud Italia e dell’articolo 33-septies del Decreto-Legge n.179/2021 è stata realizzato il Polo Strategico Nazionale (PSN), l’infrastruttura promossa dal Dipartimento per la Trasformazione Digitale che, insieme alle altre infrastrutture digitali qualificate e sicure, consente di fornire alle amministrazioni tutte le soluzioni tecnologiche adeguate e gli strumenti per realizzare il percorso di migrazione.
Il Regolamento attuativo dell’articolo 33-septies del Decreto-Legge n.179/2021 ha fissato al 28 febbraio 2023 il termine per la trasmissione dei piani di migrazione da parte delle amministrazioni.
Dopo la presentazione dei Piani di migrazione, le amministrazioni devono gestire al meglio il trasferimento in cloud di dati, servizi e applicativi. Una fase da condurre e concludere entro il 30 giugno 2026, avendo cura dei riferimenti tecnici e normativi necessari per completare una migrazione di successo.
Per realizzare al meglio il proprio piano di migrazione, le amministrazioni possono far riferimento al sito cloud.italia.it dove sono disponibili diversi strumenti a supporto, tra cui:
•
il manuale di abilitazione al cloud, che da un punto di vista tecnico accompagna le PA nel percorso che parte dall’identificazione degli applicativi da migrare in cloud fino ad arrivare alla valutazione degli indicatori di risultato a migrazione avvenuta;
•
un framework di lavoro che descrive il modello organizzativo delle unità operative (unità di controllo, unità di esecuzione e centri di competenza) che eseguiranno il programma di abilitazione;
•
articoli tecnici di approfondimento relativi ai principali aspetti da tenere in considerazione durante una migrazione al cloud.
In particolare, mediante l'accesso agli strumenti sopra citati le amministrazioni possono trovare suggerimenti utili riguardo ai seguenti temi:
•
come riconoscere e gestire possibili situazioni di lock-in;
•
raccomandazioni sugli aspetti legati al back up dei dati e al disaster recovery;
•
consigli sulla scelta della migliore strategia di migrazione dal re-host al re-architect in base alle caratteristiche degli applicativi da migrare;
•
come migliorare la migrazione in cloud grazie a un approccio DevOps;
•
come definire e separare correttamente i ruoli tra Unità di Controllo (chi governa il progetto di migrazione) e Unità di esecuzione (chi realizza la migrazione);
•
come misurare costi/benefici derivanti dalla migrazione;
•
come stabilire un perimetro di responsabilità condivise tra amministrazione utente e fornitore di servizi cloud;
•
come sfruttare al massimo le opportunità del cloud grazie alle applicazioni cloud native, al re-architect e al re-purchase.
In caso di disponibilità all’interno del Catalogo dei servizi cloud per la PA qualificati da ACN di una soluzione SaaS che risponda alle esigenze delle amministrazioni, è opportuno valutare la migrazione verso il SaaS come soluzione prioritaria (principio SaaS-first) rispetto alle altre tipologie IaaS e PaaS.
95
Quindi, anche al fine di riqualificare la spesa della PA in sviluppo e manutenzione applicativa, le amministrazioni possono promuovere anche iniziative per la realizzazione di applicativi cloud native da erogare come SaaS mediante accordi verso altre amministrazioni anche attraverso il riuso di codice disponibile sul catalogo Developers Italia, nel rispetto della normativa applicabile.
Altro aspetto da curare è quello dei costi operativi correnti. Con la migrazione al cloud, ci sono grandi opportunità di risparmio economico, ma occorre strutturarsi per una corretta gestione dei costi cloud, sia da un punto di vista contrattuale che tecnologico.
Inoltre, con il crescere di servizi digitali forniti ad uno stesso ente da una molteplicità di fornitori diversi, anche via cloud, cresce notevolmente la complessità della gestione del parco applicativo, rendendo difficile la concreta integrazione tra i software dell’ente, l’effettiva possibilità di interoperabilità verso altri enti, la corretta gestione dei dati, ecc. Questo richiede all’Ufficio RTD, in forma singola o associata, l'evoluzione verso nuove architetture a “micro-servizi”.
Lo stesso concetto di “Sistema Pubblico di Connettività” (SPC), ancora presente nel CAD all’art.73, dovrà trovare una sua evoluzione basato sulla nuova logica cloud. Oggi è proprio il cloud computing, con la sua natura decentrata, policentrica e federata, a rendere possibile il disegno originario del SPC e salvaguardare pienamente l'autonomia degli enti, la neutralità tecnologica e la concorrenza sulle soluzioni ICT destinate alle PA.
Accanto agli aspetti di natura organizzativa è necessario porre attenzione anche ad una serie di elementi di natura più tecnologica.
Lo sviluppo delle infrastrutture digitali, infatti, è parte integrante della strategia di modernizzazione del settore pubblico: esse devono essere affidabili, sicure, energeticamente efficienti ed economicamente sostenibili e garantire l’erogazione di servizi essenziali per il Paese.
L’evoluzione tecnologica espone, tuttavia, i sistemi a nuovi e diversi rischi, anche con riguardo alla tutela dei dati personali. L’obiettivo di garantire una maggiore efficienza dei sistemi non può essere disgiunto dall’obiettivo di garantire contestualmente un elevato livello di sicurezza delle reti e dei sistemi informativi utilizzati dalla Pubblica Amministrazione.
Tuttavia, come già rilevato a suo tempo da AGID attraverso il Censimento del Patrimonio ICT della PA, molte infrastrutture della PA risultano prive dei requisiti di sicurezza e di affidabilità necessari e, inoltre, sono carenti sotto il profilo strutturale e organizzativo. Ciò espone il Paese a numerosi rischi, tra cui quello di interruzione o indisponibilità dei servizi e quello di attacchi cyber, con conseguente accesso illegittimo da parte di terzi a dati (o flussi di dati) particolarmente sensibili o perdita e alterazione degli stessi dati.
Lo scenario delineato pone l’esigenza immediata di attuare un percorso di razionalizzazione delle infrastrutture per garantire la sicurezza dei servizi oggi erogati tramite infrastrutture classificate come gruppo B, mediante la migrazione degli stessi verso infrastrutture conformi a standard di qualità, sicurezza, performance e scalabilità, portabilità e interoperabilità.
Con il presente documento, in coerenza con quanto stabilito dall’articolo 33-septies del decreto-legge 18 ottobre 2012, n. 179, si ribadisce che:
96
●
con riferimento alla classificazione dei data center di cui alla Circolare AGID 1/2019 e ai fini della strategia di razionalizzazione dei data center, le categorie “infrastrutture candidabili ad essere utilizzate da parte dei PSN” e “Gruppo A” sono rinominate “A”;
●
al fine di tutelare l'autonomia tecnologica del Paese, consolidare e mettere in sicurezza le infrastrutture digitali delle pubbliche amministrazioni di cui all'articolo 2, comma 2, lettere a) e c) del decreto legislativo 7 marzo 2005, n. 82, garantendo, al contempo, la qualità, la sicurezza, la scalabilità, l’efficienza energetica, la sostenibilità economica e la continuità operativa dei sistemi e dei servizi digitali, il Dipartimento per la Trasformazione Digitale della Presidenza del Consiglio dei Ministri promuove lo sviluppo di un’infrastruttura ad alta affidabilità localizzata sul territorio nazionale, anche detta Polo Strategico Nazionale (PSN), per la razionalizzazione e il consolidamento dei Centri per l'elaborazione delle informazioni (CED) destinata a tutte le pubbliche amministrazioni;
●
le amministrazioni centrali individuate ai sensi dell'articolo 1, comma 3, della legge 31 dicembre 2009, n. 196, nel rispetto dei principi di efficienza, efficacia ed economicità dell'azione amministrativa, migrano i loro Centri per l'elaborazione delle informazioni (CED) e i relativi sistemi informatici, privi dei requisiti fissati dalla Circolare AGID 1/2019 e, successivamente, dal regolamento di cui all’articolo 33-septies, comma 4, del decreto-legge 18 ottobre 2012, n. 179 (di seguito Regolamento cloud e infrastrutture), verso l’infrastruttura del PSN o verso altra infrastruttura propria già esistente e in possesso dei requisiti fissati dalla Circolare AGID 1/2019 e, successivamente, dal Regolamento cloud e infrastrutture. Le amministrazioni centrali, in alternativa, possono migrare i propri servizi verso soluzioni cloud qualificate, nel rispetto di quanto previsto dalle Circolari AGID n. 2 e n. 3 del 2018 e, successivamente, dal Regolamento cloud e infrastrutture;
●
le amministrazioni locali individuate ai sensi dell'articolo 1, comma 3, della legge 31 dicembre 2009, n. 196, nel rispetto dei principi di efficienza, efficacia ed economicità dell’azione amministrativa, migrano i loro Centri per l'elaborazione delle informazioni (CED) e i relativi sistemi informatici, privi dei requisiti fissati dalla Circolare AGID 1/2019 e, successivamente, dal regolamento cloud e infrastrutture, verso l'infrastruttura PSN o verso altra infrastruttura della PA già esistente in possesso dei requisiti fissati dallo stesso Regolamento cloud e infrastrutture. Le amministrazioni locali, in alternativa, possono migrare i propri servizi verso soluzioni cloud qualificate nel rispetto di quanto previsto dalle Circolari AGID n. 2 e n. 3 del 2018 e, successivamente, dal Regolamento cloud e infrastrutture;
●
le amministrazioni non possono investire nella costruzione di nuovi data center per ridurre la frammentazione delle risorse e la proliferazione incontrollata di infrastrutture con conseguente moltiplicazione dei costi. È ammesso il consolidamento dei data center nel rispetto di quanto previsto dall'articolo 33-septies del Decreto-legge 179/2012 e dal Regolamento di cui al comma 4 del citato articolo 33-septies.
Nel delineare il processo di razionalizzazione delle infrastrutture è necessario far riferimento anche a quanto previsto dalla “Strategia Cloud Italia”. In tal senso il documento prevede:
i) la creazione del PSN, la cui gestione e controllo di indirizzo siano autonomi da fornitori extra UE, destinato ad ospitare sul territorio nazionale principalmente dati e servizi strategici la cui compromissione può avere un impatto sulla sicurezza nazionale, in linea con quanto previsto
97
in materia di perimetro di sicurezza nazionale cibernetica dal Decreto-legge 21 settembre 2019, n. 105 e dal DPCM 81/2021;
ii) un percorso di qualificazione dei fornitori di cloud pubblico e dei loro servizi per garantire che le caratteristiche e i livelli di servizio dichiarati siano in linea con i requisiti necessari di sicurezza, affidabilità e rispetto delle normative rilevanti e iii) lo sviluppo di una metodologia di classificazione dei dati e dei servizi gestiti dalle pubbliche amministrazioni, per permettere una migrazione di questi verso la soluzione cloud più opportuna (PSN o adeguata tipologia di cloud qualificato).
Con riferimento al punto i) creazione del PSN, a dicembre 2022, in coerenza con la relativa milestone PNRR associata, è stata realizzata e testata l’infrastruttura PSN. Si ricorda che tale infrastruttura eroga servizi professionali di migrazione verso l’infrastruttura PSN, servizi di housing, hosting e cloud nelle tipologie IaaS, PaaS.
Per maggiori informazioni sui servizi offerti da PSN si rimanda alla convenzione pubblicata sul sito della Presidenza del Consiglio dei Ministri.
Nel 2023 sono stati pubblicati e conclusi tre avvisi per la migrazione verso il PSN a valere sulla misura 1.1 del PNRR che hanno visto l’adesione di oltre 300 tra amministrazioni centrali e aziende sanitarie locali e ospedaliere. Per quanto riguarda le ASL/AO, in particolare, è stata offerta l’opportunità di decidere la destinazione dei propri servizi tra PSN, Infrastrutture della PA adeguate e soluzioni cloud qualificate coerentemente con quanto disposto dall’articolo 33-septies del Decreto-legge 179/2012. 130 Aziende sanitarie hanno scelto di portare almeno parte dei propri servizi presso il PSN.
Con riferimento ai punti ii) qualificazione e iii) classificazione a dicembre 2021 sono stati pubblicati il Regolamento cloud e infrastrutture e a gennaio 2022 i relativi atti successivi. A febbraio e a luglio sono stati pubblicati ulteriori Decreti ACN ed è prevista la pubblicazione da parte di ACN di un nuovo Regolamento.
Con riferimento alla misura 1.2 del PNRR a marzo 2023 sono stati raccolti e ammessi a finanziamento più dei 12.464 piani di migrazione richiesti dal target è stato raggiunto e superato il target italiano previsto per settembre 2023 con la migrazione di oltre 1.100 enti locali che hanno migrato i loro servizi verso soluzioni cloud qualificate.
Con riferimento al tema del cloud federato, si premette che la definizione tecnica coerentemente con la ISO/IEC 22123-1:2023 è la seguente: "modello di erogazione di servizi cloud forniti da 2 o più cloud service provider che si uniscono mediante un accordo che preveda un insieme concordato di procedure, processi e regole comuni finalizzato all'erogazione di servizi cloud". Le amministrazioni con infrastrutture classificate "A" che hanno deciso di investire sui propri data center per valorizzare i propri asset ai fini della razionalizzazione dei centri elaborazione dati, adeguandoli secondo le modalità e i termini previsti ai requisiti di cui al Regolamento adottato ai sensi del comma 4 dell'articolo 33-septies del Decreto-legge 179/2012 e agli atti successivi di ACN, hanno la facoltà di valutare la possibilità di stringere accordi in tal senso per raggiungere maggiori livelli di affidabilità, sicurezza ed elasticità, purché siano rispettati i princìpi di efficacia ed efficienza dell'azione amministrativa e della normativa applicabile. Le amministrazioni che dovessero stipulare tali accordi
98
realizzerebbero così le infrastrutture cloud federate della PA che si affiancano all’infrastruttura Polo Strategico Nazionale nel rispetto dell’articolo 33-septies del decreto-legge 18 ottobre 2012, n. 179.
Per “infrastrutture di prossimità” (o edge computing) si intendono i nodi periferici (edge nodes), misurati come numero di nodi di calcolo con latenze inferiori a 20 millisecondi; si può trattare di un singolo server o di un altro insieme di risorse di calcolo connesse, operati nell'ambito di un'infrastruttura di edge computing, generalmente situati all'interno di un edge data center che opera all'estremità dell'infrastruttura, e quindi fisicamente più vicini agli utenti destinatari rispetto a un nodo cloud in un data center centralizzato".
Le amministrazioni che intendono realizzare e/o utilizzare infrastrutture di prossimità verificano la conformità di queste ai requisiti del Regolamento di cui al comma 4 dell’articolo 33-septies del DL 179/2012.
Punti di attenzione e azioni essenziali per tutti gli enti
1) L’attuazione dell’art.33-septies Decreto-legge 179/2012, e del principio cloud-first, deve essere tra gli obiettivi prioritari dell’ente. Occorre curare da subito anche gli aspetti di sostenibilità economico-finanziaria nel tempo dei servizi attivati, avendo cura di verificare gli impatti della migrazione sui propri capitoli di bilancio relativamente sia ai costi correnti (OPEX) sia agli investimenti in conto capitale (CAPEX).
2) La gestione dei servizi in cloud deve essere presidiata dall’ente in tutto il ciclo di vita degli stessi e quindi è necessaria la disponibilità di competenze specialistiche all’interno dell’Ufficio RTD, in forma singola o associata.
Approfondimento tecnologico per gli RTD
1) La piena abilitazione al cloud richiede l’evoluzione del parco applicativo software verso la logica as a service delle applicazioni esistenti, andando oltre il mero lift-and-shift dei server, progettando opportuni interventi di rearchitect, replatform o repurchase per poter sfruttare le possibilità offerte oggi dalle moderne piattaforme computazionali e dagli algoritmi di intelligenza artificiale. In tal senso, occorre muovere verso architetture a “micro-servizi” le cui caratteristiche sono, in sintesi, le seguenti:
•
ogni servizio non ha dipendenze esterne da altri servizi e gestisce autonomamente i propri dati (self-contained)
•
ogni servizio comunica con l'esterno attraverso API/webservice e senza dipendenza da stati pregressi (lightweight/stateless)
•
ogni servizio può essere implementato con differenti linguaggi e tecnologie, in modo indipendente dagli altri servizi (implementation-indipendent)
•
ogni servizio può essere dispiegato in modo automatico e gestito indipendentemente dagli altri servizi (indipendently deployable)
•
ogni servizio implementa un insieme di funzioni legate a procedimenti e attività amministrative, non ha solo scopo tecnologico (business-oriented):
2) È compito dell’Ufficio RTD curare sia gli aspetti di pianificazione della migrazione/abilitazione al cloud che l’allineamento dello stesso con l'implementazione delle relative opportunità di riorganizzazione dell’ente offerte dall’abilitazione al cloud e dalle nuove architetture a micro-servizi.
99
3) La gestione del ciclo di vita dei servizi in cloud dell’amministrazione richiede la strutturazione di opportuni presidi organizzativi e strumenti tecnologici per il cloud-cost-management, in forma singola o associata.
Contesto normativo e strategico
In materia di infrastrutture esistono una serie di riferimenti sia normativi che strategici a cui le amministrazioni devono attenersi. Di seguito un elenco delle principali fonti.
Riferimenti normativi nazionali:

Il sistema pubblico di connettività
Scenario
Il Sistema Pubblico di Connettività (SPC) garantisce alle Amministrazioni aderenti sia l’interscambio di informazioni in maniere riservata che la realizzazione della propria infrastruttura di comunicazione.
A tale Sistema possono interconnettersi anche le reti regionali costituendo così una rete di comunicazione nazionale dedicato per l’interscambio di informazioni tra le pubbliche amministrazioni sia centrali che locali.
Per effetto della legge n. 87 del 3 luglio 2023, di conversione del Decreto-legge 10 maggio 2023, n. 51 la scadenza dell’attuale Contratto Quadro è stata prorogata al 31 dicembre 2024; entro questa data sarà reso disponibile alle Amministrazioni interessate il nuovo Contratto Quadro che prevederà oltre ai servizi di connettività anche i servizi di telefonia fissa come da informativa Consip del 13 Aprile 2023.
Il Sistema Pubblico di Connettività fornisce un insieme di servizi di rete che:
•
permette alla singola Pubblica Amministrazione, centrale o locale, di interconnettere le proprie sedi e realizzare così anche l’infrastruttura interna di comunicazione;
•
realizza un’infrastruttura condivisa di interscambio consentendo l’interoperabilità tra tutte le reti delle pubbliche amministrazioni salvaguardando la sicurezza dei dati;
•
garantisce l’interconnessione della Pubblica Amministrazione alla rete Internet;


Capitolo 7 - Sicurezza informatica
Sicurezza informatica
Scenario
L’evoluzione delle moderne tecnologie e la conseguente possibilità di ottimizzare lo svolgimento dei procedimenti amministrativi con l’obiettivo di rendere efficace, efficiente e più economica l’azione amministrativa, ha reso sempre più necessaria la “migrazione” verso il digitale che, però, al contempo, sta portando alla luce nuovi rischi, esponendo imprese e servizi pubblici a possibili attacchi cyber. In quest’ottica, la sicurezza e la resilienza delle reti e dei sistemi, su cui tali tecnologie poggiano, sono il baluardo necessario a garantire, nell’immediato, la sicurezza del Paese e, in prospettiva, lo sviluppo e il benessere dello Stato e dei cittadini.
La recente riforma dell’architettura nazionale cyber, attuata attraverso l’adozione del decreto-legge 14 giugno 2021, n. 82 che ha istituito l’Agenzia per la Cybersicurezza Nazionale (ACN), ha come obiettivo, tra gli altri, quello di sviluppare e rafforzare le capacità cyber nazionali, garantendo l’unicità istituzionale di indirizzo e azione, anche mediante la redazione e l’implementazione della Strategia nazionale di cybersicurezza, che considera cruciale, per il corretto “funzionamento” del sistema Paese, la sicurezza dell’ecosistema digitale alla base dei servizi erogati dalla Pubblica Amministrazione, con specifica attenzione ai beni ICT. Tali beni supportano le funzioni e i servizi essenziali dello Stato e, purtroppo, come dimostrano gli ultimi rapporti di settore, sono tra i bersagli preferiti degli attacchi cyber.
Per garantire lo sviluppo e il rafforzamento delle capacità cyber nazionali, con il Piano Nazionale di Ripresa e Resilienza e con i Fondi per l’attuazione e la gestione della Strategia nazionale di cybersicurezza sono state destinate significative risorse alla sicurezza cibernetica e alle misure tese a realizzare un percorso di miglioramento della postura di sicurezza del sistema Paese nel suo insieme e, in particolare, della Pubblica Amministrazione.
Gli obiettivi e i risultati attesi, definiti successivamente nel presente capitolo, sono in linea con specifici interventi realizzati dall’ACN in favore delle pubbliche amministrazioni per cui sono state individuate specifiche aree di miglioramento. In particolare, il riferimento è alla necessità di:
•
prevedere dei modelli di gestione centralizzati della cybersicurezza, coerentemente con il ruolo trasversale associato (obiettivo 7.1 di questo Piano);
•
definire processi di gestione e mitigazione del rischio cyber, sia interni sia legati alla gestione delle terze parti di processi IT (obiettivi 7.2, 7.3, 7.4);
•
promuovere attività legate al miglioramento della cultura cyber delle Amministrazioni (obiettivo 7.5).
All’interno di questo contesto, AGID metterà a disposizione della Pubblica Amministrazione una serie di piattaforme e di servizi, che verranno erogati tramite il proprio CERT, finalizzati alla conoscenza e al contrasto dei rischi cyber legati al patrimonio ICT della PA (obiettivo 7.6)
	
\end{document}