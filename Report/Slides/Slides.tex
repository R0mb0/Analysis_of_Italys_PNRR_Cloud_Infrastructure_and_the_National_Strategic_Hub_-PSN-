\documentclass[hidelinks,aspectratio=169]{beamer}
\usepackage[italian]{babel} 
\usepackage[utf8]{inputenc} 
\usepackage{fourier} 

%Slide colors
\usetheme{Madrid}
%\usecolortheme{beaver}

% Images
\usepackage{graphicx}
\usepackage{caption}
\usepackage{subcaption}
\usepackage{float}
\graphicspath{{Images}}

% Stop hyphenation
\usepackage[none]{hyphenat}

% Minipages in the same line
\usepackage{tabularx}

% Coloring links
\usepackage{xcolor}

% Enumerate abc
\usepackage{enumerate}

% Redefines caption setup in way to remove "Figure:"
\usepackage{caption}
\captionsetup[figure]{labelformat=empty}

% License
\usepackage[
type={CC},
modifier={by-nc-sa},
version={4.0},
]{doclicense}

%------------------- Commands zone --------------------

%Command to zoom in
\usepackage{mwe}
\makeatletter
\newsavebox\zb@x
\newcounter{z@@m}
\usepackage{calc}
\newdimen\B@r\newdimen\P@r
\newdimen\@zw\newdimen\@zh\newdimen\@zd

\newcommand{\zoombox}[2][0]{%
	\leavevmode%
	\sbox\zb@x{#2}%
	\setlength\B@r{1pt*\ratio{\wd\zb@x}{\ht\zb@x+\dp\zb@x}}%
	\setlength\P@r{1pt*\ratio{\paperwidth}{\paperheight}}%
	\ifdim\B@r>\P@r\relax%
	\setlength\@zw{\wd\zb@x}\setlength\@zh{\@zw*\ratio{\paperheight}{\paperwidth}}%
	\setlength\@zd{(\@zh-\ht\zb@x-\dp\zb@x)*\real{0.5}+\dp\zb@x}%
	\setlength\@zh{\@zh-\@zd}%
	\else%
	\setlength\@zh{\ht\zb@x+\dp\zb@x}%
	\setlength\@zw{\@zh*\ratio{\paperwidth}{\paperheight}}%
	\setlength\@zh{\ht\zb@x}\setlength\@zd{\dp\zb@x}%
	\fi%
	\makebox[0pt][l]{\makebox[\wd\zb@x][c]{\makebox[\@zw][l]{%
				\pdfdest name {zbfs\thez@@m} fitr
				width  \@zw\space
				height \@zh\space
				depth  \@zd\space
	}}}%
	\pdfdest name {zb\thez@@m} fitr
	width  \wd\zb@x\space
	height \ht\zb@x\space
	depth  \dp\zb@x\space
	\immediate\pdfannot 
	width  \wd\zb@x\space
	height \ht\zb@x\space
	depth  \dp\zb@x\space
	{%
		/Subtype/Link/H/N
		/Border [0 0 #1 [1 2]]
		/A <<
		/S/JavaScript
		/JS (
		if(typeof(zoomed)=='undefined'||!zoomed){
			var lastView=this.viewState;
			if(app.fs.isFullScreen) this.gotoNamedDest('zbfs\thez@@m');
			else this.gotoNamedDest('zb\thez@@m');
			zoomed=true;
		}else{
			this.viewState=lastView;
			zoomed=false;
		}
		)
		>>
	}%
	\usebox{\zb@x}%
	\stepcounter{z@@m}%
} 
\makeatother

%------------------- Header --------------------
\title[	PNRR, PSN e PDND]{\small \textbf{PNRR, PSN e PDND}}
\author[Francesco Rombaldoni]{}
\date{Anno Accademico 2025/2026}

\begin{document}
	
	\begin{frame}
		\vspace*{-5mm}
		\begin{center}
			\hspace*{30mm}\zoombox{\includegraphics[scale=0.2]{logo-uniurb-2016.jpg}}
			\vspace*{2mm}
			\newline
			{\Large UNIVERSITÀ DEGLI STUDI DI URBINO CARLO BO}\\
			\vspace*{0.5mm}
			Dipartimento di Scienze Pure e Applicate\\
			\vspace*{0.5mm}
			Corso di Laurea in Informatica e Innovazione Digitale\\
			\hspace*{10mm}\noindent\rule{110mm}{0.4pt}\newline
			\vspace*{0.5mm}
		Presentazione progetto programmazione per \textbf{Sistemi Distribuiti}\\
		\vspace*{5mm}
		\textbf{\large {Trasformazione digitale della Pubblica Amministrazione italiana: \\
				PNRR, Polo Strategico Nazionale e Piattaforma Digitale Nazionale Dati}}
		\end{center}
	\end{frame}
	
	\begin{frame}
		\centering
		\fboxrule=2pt
		\fbox
		{
			\begin{minipage}{0.9\linewidth}
				\small{Il seguente documento è ottimizzato per la visualizzazione digitale con \href{https://get.adobe.com/it/reader/}{\textcolor{blue}{Adobe~Acrobat~Reader}}.}  
			\end{minipage}
		}
	\end{frame}
	
	\begin{frame}
		\tableofcontents
	\end{frame}
	
	\section{Introduzione}
	
	\begin{frame}{Introduzione}
		\begin{itemize}
			\item Negli ultimi anni, la digitalizzazione dei servizi pubblici è passata da una somma di iniziative locali a un percorso più coordinato a livello nazionale.
		\end{itemize}
		\begin{center}
			\zoombox{\includegraphics[scale=0.5]{ecosistema.png}}
		\end{center}
	\end{frame}
	
	\section{Prima della transizione}
	\begin{frame}{Prima della transizione}
		\begin{itemize}
			\item Soluzioni applicative e infrastrutturali sono nate in tempi diversi, con tecnologie diverse e con livelli di maturità differenti
			\item Le modalità di scambio di dati e informazioni tra enti sono state spesso il risultato di accordi bilaterali e integrazioni specifiche.
			\item la gestione della sicurezza, del monitoraggio, della continuità operativa e dell'aggiornamento infrastrutturale era risultata disomogenea.
		\end{itemize}
	\end{frame}
	
	\section{PNRR e governance}
	\begin{frame}{PNRR e governance}
		\begin{itemize}
			\item Il PNRR agisce sia come fonte di finanziamento sia come meccanismo di indirizzo: obiettivi, milestone, target e misure di controllo trasformano la digitalizzazione in un programma coordinato e monitorabile.
		\end{itemize}
	\end{frame}
	
	\section{Il Piano Triennale 2024-2026}
	\begin{frame}{Il Piano Triennale 2024-2026}
		\begin{itemize}
			\item Il \emph{Piano Triennale per l’informatica nella PA 2024--2026} descrive un impianto che collega la trasformazione dei servizi alla necessità di un framework uniforme.
			\item La \textbf{PDND} (Piattaforma Digitale Nazionale Dati) è presentata come il \emph{layer focale} per la condivisione di dati e processi tra amministrazioni, con l’obiettivo di attuare il principio \emph{once-only}.
		\end{itemize}
	\end{frame}
	
	\section{Infrastrutture e cloud}
	\begin{frame}{Infrastrutture e cloud}
		\begin{itemize}
			\item La sezione infrastrutture del Piano Triennale inquadra la strategia \textbf{Cloud Italia} come un percorso attuativo che risponde a tre sfide: autonomia tecnologica, controllo sui dati e resilienza dei servizi.
			\item 	Il \textbf{PSN} (Polo Strategico Nazionale) viene descritto nel Piano Triennale come infrastruttura ad alta affidabilità localizzata sul territorio nazionale, promossa per la razionalizzazione e il consolidamento dei Centri di Elaborazione Dati (CED) della PA, garantendo qualità, sicurezza, scalabilità, sostenibilità economica e continuità operativa.
			\item Il Piano richiama il \textbf{SPC} (Sistema Pubblico di Connettività) come rete e insieme di servizi di connettività che consente l’interscambio riservato di informazioni, l’interoperabilità tra reti delle amministrazioni e l’interconnessione a Internet.
		\end{itemize}
	\end{frame}
	
	\section{PSN 2025: servizi, sovranità, resilienza e modelli di erogazione}
	\begin{frame}{PSN 2025: servizi, sovranità, resilienza e modelli di erogazione}
		\begin{center}
			\textbf{Gestione PSN (Oracle Alloy)}
		\end{center}
		\begin{itemize}
			\item fault domain (con ridondanza e anti-affinity),
			\item replica storage,
			\item replica database (Oracle RAC),
			\item assenza di downtime programmato per maintenance,
			\item disponibilità di più region (Nord e Sud) e possibilità di disaster recovery tra region con RPO definiti in convenzione.
		\end{itemize}
	\end{frame}
	
	\subsection{Secure Public Cloud}
	\begin{frame}{Secure Public Cloud}
		\begin{center}
			concezione del \textbf{Secure Public Cloud} come accesso a region pubbliche di hyperscaler selezionati, con l’aggiunta di elementi di sicurezza e governance erogati dai data center PSN.
		\end{center}
		\begin{itemize}
			\item gestione chiavi di crittografia esterna al controllo del cloud provider;
			\item modello di governance e ambienti segregati standard;
			\item controlli su traffico (hub \& spoke) e monitoraggio;
			\item backup conservati nel perimetro PSN;
			\item scenari di confidential computing (in particolare nella variante Confidential Azure) con meccanismi aggiuntivi di trasparenza e controllo.
		\end{itemize}
	\end{frame}
	
	\section{Sicurezza e gestione del rischio}
	\begin{frame}{Sicurezza e gestione del rischio}
		 Viene richiamata la riforma dell’architettura nazionale cyber con la nascita dell’\textbf{ACN} (Agenzia per la Cybersicurezza Nazionale), e viene indicata la necessità di:
		\begin{itemize}
			\item modelli di gestione centralizzati della cybersicurezza;
			\item processi di gestione e mitigazione del rischio cyber, inclusa gestione delle terze parti;
			\item promozione della cultura cyber;
			\item servizi e piattaforme (es. CERT) a supporto della PA.
		\end{itemize}
	\end{frame}
	
	\section{Caso di studio operativo}
	\begin{frame}{Caso di studio operativo}
		Negli estratti del protocollo AGENAS relativi ai servizi Sogei, compaiono elementi tipici di una gestione ICT industrializzata:
		\begin{itemize}
			\item \textbf{Disaster recovery}: replica asincrona, verifiche semestrali e prove annuali con utenti selezionati;
			\item \textbf{SOC (Security Operation Center)}: centralizzazione log via SIEM, monitoraggio real-time, incident response, patch management, reportistica;
			\item \textbf{IAM (Identity and Access Management)}: gestione identità/autorizzazioni e componenti di certificazione.
		\end{itemize}
	\end{frame}
	
	\section{in che senso il modello tende alla centralizzazione?}
	\begin{frame}{in che senso il modello tende alla centralizzazione?}
		\begin{itemize}
			\item la PDND come layer nazionale di regole e catalogo per l’interoperabilità (accordi, autorizzazioni, tracciamento);
			\item la diffusione e il consolidamento di piattaforme nazionali per identità, pagamenti, notifiche e basi dati di interesse nazionale;
			\item il PSN come infrastruttura di riferimento per la migrazione di dati/servizi e come volano per la razionalizzazione dei CED.
		\end{itemize}
	\end{frame}
	
	\section{Ibridazione}
	\begin{frame}{Ibridazione}
		\begin{itemize}
			\item cloud e multi-region con replica e DR;
			\item coesistenza di PSN Managed e Secure Public Cloud con hyperscaler;
			\item possibilità di infrastrutture cloud federate della PA e concetti di edge/infrastrutture di prossimità.
		\end{itemize}
	\end{frame}
	
	\section{Concentrazione del rischio}
	\begin{frame}{Concentrazione del rischio}
		\begin{itemize}
			\item Non è necessario che tutto stia fisicamente in un unico luogo: basta che esistano punti di controllo logico affinché un attacco ben riuscito possa avere un \emph{blast radius} maggiore
			\item L’incentivo ad attacchi mirati aumenta.
		\end{itemize}
	\end{frame}
	
	\section{Il fattore umano}
	\begin{frame}{Il fattore umano}
		Questa architettura fa pensare che il fattore umano rimane l’elemento più vulnerabile.
		\begin{itemize}
			\item aumenta il numero di attori coinvolti (enti, fornitori, subfornitori, operatori);
			\item aumentano i privilegi necessari per operare (gestione tenant, gestione chiavi, emergenze);
			\item non esistono controlli stringenti su accessi amministrativi, segregazione dei compiti e audit.
		\end{itemize}
	\end{frame}
	
	\section{dipendenza operativa e qualità reale di un SOC}
	\begin{frame}{dipendenza operativa e qualità reale di un SOC}
		\begin{center}
			\textbf{Un SOC esiste davvero quando:}
		\end{center}
		\begin{itemize}
				\item riceve log completi e corretti (non solo infrastruttura, anche applicazioni);
			\item correla e genera alert utili (riduzione falsi positivi / falsi negativi);
			\item ha processi di incident response e canali decisionali rapidi;
			\item esegue esercitazioni e post-mortem che producono miglioramenti.
		\end{itemize}
	\end{frame}
	
	\section{Interoperabilità}
	\begin{frame}{Interoperabilità}
		La PDND introduce un catalogo e regole comuni. Tuttavia, l’interoperabilità può fallire anche se ``l’API esiste'', quando:
		\begin{itemize}
			\item la semantica dei dati è ambigua o incoerente tra enti;
			\item mancano regole chiare di versioning e deprecazione;
			\item il dato fonte cambia senza un meccanismo affidabile di notifica/aggiornamento;
			\item i fruitori implementano cache o repliche senza governance.
		\end{itemize}
	\end{frame}
	
	\section{Resilienza e disponibilità}
	\begin{frame}{Resilienza e disponibilità}
		\begin{center}
			\textbf{La resilienza promessa dal cloud e dal PSN dipende da come vengono configurati i sistemi applicativi.}
		\end{center}
		\begin{itemize}
			\item un’applicazione può essere ospitata in cloud ma restare fragile (dipendenze esterne, configurazioni errate, single points of failure logici);
			\item un piano di DR senza prove periodiche può fallire nel momento del bisogno;
			\item la gestione dei backup può essere inefficace se non verificata (restore test, integrity, tempi).
		\end{itemize}
	\end{frame}
	
	\section{Lock-in non solo tecnologico}
	\begin{frame}{Lock-in non solo tecnologico}
		\begin{center}
			\textbf{È un obiettivo realistico solo se si riconosce che il lock-in ha almeno tre componenti:}
		\end{center}
		\begin{itemize}
			\item \textbf{tecnologica}: uso di servizi proprietari difficili da migrare;
			\item \textbf{operativa}: strumenti e runbook costruiti intorno a una piattaforma;
			\item \textbf{organizzativa}: competenze concentrate su uno stack, difficili da sostituire.
		\end{itemize}
	\end{frame}
	
	\section{Condizioni di successo}
	\begin{frame}{Condizioni di successo}
		\begin{center}
			\textbf{Perché i benefici attesi siano realmente ottenuti, alcune condizioni appaiono essenziali:}
		\end{center}
		\begin{itemize}
			\item \textbf{Disciplina sugli accessi}: identità e privilegi devono essere governati con rigore, audit e segregazione dei compiti.
			\item \textbf{Osservabilità e audit end-to-end}: logging e monitoring non solo infrastrutturali ma anche applicativi, con procedure di risposta agli incidenti.
			\item \textbf{DR verificato}: backup e disaster recovery devono essere testati periodicamente e non solo dichiarati.
			\item \textbf{Interoperabilità come contratto}: API e e-service devono avere versioning, semantica chiara e gestione dell’evoluzione.
			\item \textbf{Gestione fornitori}: controllo delle terze parti come parte integrante della postura di sicurezza.
		\end{itemize}
	\end{frame}
	
\end{document}