\documentclass[12pt]{article}

\usepackage[utf8]{inputenc}
\usepackage[T1]{fontenc}
\usepackage[italian]{babel}

\usepackage{amsmath,amssymb}
\usepackage{parskip}
\usepackage[margin=1in]{geometry}
\usepackage[svgnames]{xcolor}
\usepackage{graphicx}
\usepackage{array}
\usepackage{enumitem}
\usepackage{appendix}
\usepackage{fancyhdr}
\usepackage{titling}
\usepackage{authblk}
\usepackage{hyperref}

\usepackage[many]{tcolorbox}

% Nel template originale è presente minted: lo mantengo per coerenza.
\usepackage{minted}
\setminted{
	frame=single,
	linenos,
	fontsize=\footnotesize,
	breaklines,
	breakanywhere,
	numbersep=4pt,
	xleftmargin=6pt,
	framesep=2pt
}

\usepackage{csquotes}
\usepackage{biblatex}
\addbibresource{riferimenti.bib}

\hypersetup{
	colorlinks=true,
	linkcolor=MidnightBlue,
	urlcolor=MidnightBlue,
	citecolor=MidnightBlue
}

\title{\color{Black}\bf{Trasformazione digitale della Pubblica Amministrazione italiana: \\
		PNRR, Polo Strategico Nazionale e Piattaforma Digitale Nazionale Dati}}
\author[1]{\color{Black}\bf{Francesco Rombaldoni}}
\affil[1]{\texttt{f.rombaldoni@campus.uniurb.it}}
\date{}

\begin{document}
	
	\fancypagestyle{firstpage}
	{
		\fancyhead[L]{\footnotesize{\bf{Universit\`a degli Studi di Urbino Carlo Bo}}}
		\fancyhead[R]{\footnotesize{\bf{CdL Magistrale Informatica e Innovazione Digitale}}}
	}
	\thispagestyle{firstpage}
	
	\pagestyle{fancy}
	\fancyhead{}
	\fancyhead[L]{\footnotesize{\thetitle}}
	\fancyfoot{}
	\fancyfoot[R]{\footnotesize{\bf{\thepage}}}
	\fancyfoot[L]{\footnotesize{Progetto corso Sistemi Distribuiti}}
	
	\maketitle
	
	\begin{tcolorbox}[colback=WhiteSmoke, width=\linewidth, arc=1mm, auto outer arc]
		\section*{Riassunto}
		Il Piano Nazionale di Ripresa e Resilienza (PNRR) ha accelerato una trasformazione già in corso nella Pubblica Amministrazione (PA) italiana, spingendo verso un modello più uniforme e governato di gestione dei servizi digitali, dei dati e delle infrastrutture. Dai documenti rilasciati emergono due direttrici principali: da un lato, la razionalizzazione e la messa in sicurezza dell’infrastruttura tramite il Polo Strategico Nazionale (PSN) e, più in generale, la strategia \emph{cloud-first}; dall’altro, la costruzione di un sistema di interoperabilità tra enti basato sulla Piattaforma Digitale Nazionale Dati (PDND), incentrata su e-service e API.
		Il disegno complessivo tende a ridurre la frammentazione storica della PA attraverso piattaforme comuni, regole condivise e presidi di sicurezza più strutturati. Allo stesso tempo, la realizzazione tecnica rimane necessariamente distribuita (cloud, multi-region, replica, architetture a microservizi), producendo un assetto che può essere descritto come \emph{ibrido}: più centralizzato sul piano della governance e delle piattaforme abilitanti, ma distribuito sul piano infrastrutturale e applicativo.
	\end{tcolorbox}
	
	\section{Introduzione}
	Negli ultimi anni, la digitalizzazione dei servizi pubblici è passata da una somma di iniziative locali a un percorso più coordinato a livello nazionale. Questa transizione ha motivazioni esplicite: migliorare qualità e accessibilità dei servizi, ridurre duplicazioni, aumentare resilienza e sicurezza, e valorizzare il patrimonio informativo pubblico. Tuttavia, una trasformazione di questa portata produce anche una riorganizzazione di responsabilità, procedure, tecnologie e punti di controllo.
	
	\subsection{Una direzione di centralizzazione, ma con attuazione ibrida}
	Dalla ricostruzione fornita dai documenti emerge un’idea guida: storicamente la PA italiana è stata un insieme di sistemi con molte autonomie locali e con interoperabilità spesso costruita \emph{a coppie} tra enti; l’assetto desiderato dal PNRR, dai programmi collegati al cloud e dalle piattaforme nazionali tende invece a introdurre regole comuni, strumenti condivisi e un impianto infrastrutturale più razionalizzato. Questo produce una spinta verso una \emph{centralizzazione} intesa soprattutto come centralità della governance e delle piattaforme abilitanti.
	
	Tale centralizzazione, però, non coincide con un singolo ``centro'' tecnico unico: l’infrastruttura proposta è cloud, include concetti come region e fault domain, replica e disaster recovery; inoltre il disegno prevede anche elementi federati (ad esempio il richiamo a cloud federati, o la coesistenza di più modelli di erogazione). Il risultato è quindi \emph{ibrido}: maggiore coordinamento e standardizzazione, con un’esecuzione su architetture distribuite.
	
	\section{Scenario pre-transizione: frammentazione e interoperabilità non uniforme}
	La ricostruzione descrive un quadro in cui, nel limite delle norme, ogni ente della PA ha operato come un sistema relativamente indipendente. Questo significa che:
	\begin{itemize}[leftmargin=*]
		\item le soluzioni applicative e infrastrutturali sono nate in tempi diversi, con tecnologie diverse e con livelli di maturità differenti;
		\item le modalità di scambio di dati e informazioni tra enti sono state spesso il risultato di accordi bilaterali e integrazioni specifiche;
		\item la gestione della sicurezza, del monitoraggio, della continuità operativa e dell’aggiornamento infrastrutturale è risultata disomogenea.
	\end{itemize}
	
	Questo tipo di frammentazione produce conseguenze pratiche. La prima è l’inefficienza: procedure simili vengono implementate più volte, con costi ripetuti e qualità variabile. La seconda è la fragilità: un sistema complesso composto da molte ``isole'' si integra con difficoltà e spesso non riesce a offrire servizi end-to-end coerenti. La terza è la sicurezza: la postura cyber è tanto forte quanto il suo anello più debole, e in un mosaico di enti con livelli diversi diventa difficile fissare standard e verificarne l’applicazione.
	
	In questo contesto, il Piano Triennale osserva che molte infrastrutture risultano carenti sotto il profilo strutturale e organizzativo e prive di requisiti adeguati di sicurezza e affidabilità, con rischi di indisponibilità dei servizi e rischi cyber \cite{piano-triennale-2024-2026}. La necessità di intervenire non è quindi descritta solo come modernizzazione, ma come riduzione di rischi già presenti.
	
	\section{PNRR e governance: perché la trasformazione è anche organizzativa}
	Il PNRR agisce sia come fonte di finanziamento sia come meccanismo di indirizzo: obiettivi, milestone, target e misure di controllo trasformano la digitalizzazione in un programma coordinato e monitorabile \cite{pnrr-sigeco-dtd-v5}. Questo aspetto è importante perché cambia la dinamica tipica degli interventi ``a progetto'' isolati: la migrazione al cloud, l’adozione di piattaforme nazionali e la costruzione dell’interoperabilità diventano parte di un percorso che deve essere pianificato e rendicontato.
	
	In altri termini, anche quando un protocollo non descrive direttamente una tecnologia, può incidere sulla tecnologia nella misura in cui definisce vincoli, priorità, responsabilità e tempi. È una forma di centralizzazione indiretta: non impone un unico software, ma spinge verso scelte compatibili con la strategia nazionale (cloud-first, interoperabilità via piattaforma, standard condivisi) e scoraggia investimenti che aumentano la frammentazione.
	
	\section{Il Piano Triennale 2024--2026: servizi, interoperabilità e piattaforme nazionali}
	Il \emph{Piano Triennale per l’informatica nella PA 2024--2026} descrive un impianto che collega la trasformazione dei servizi alla necessità di un framework uniforme. In particolare, nella parte dedicata ai servizi, viene promossa l’architettura a microservizi come soluzione agile e scalabile, capace di standardizzare processi e facilitare il change management \cite{piano-triennale-2024-2026}. L’enfasi non è solo tecnologica: viene esplicitato che la transizione richiede formazione continua, coinvolgimento degli stakeholder, monitoraggio dell’impatto e comunicazione efficace.
	
	\subsection{Microservizi e riuso: una promessa e una condizione}
	Il Piano sottolinea che per gli enti con scarse risorse o know-how, architetture modulari e riuso di servizi sviluppati da altri enti possono ridurre il gap e velocizzare la digitalizzazione \cite{piano-triennale-2024-2026}. Questa idea ha una conseguenza diretta sul modello complessivo: se i servizi sono componibili e riusabili, serve un modo per scoprirli, accedervi e regolare l’accesso in maniera standard. In altre parole, l’approccio ``a microservizi'' diventa credibile a scala nazionale solo se esiste un livello di interoperabilità governato.
	
	\subsection{PDND: Piattaforma Digitale Nazionale Dati}
	La \textbf{PDND} (Piattaforma Digitale Nazionale Dati) è presentata come il \emph{layer focale} per la condivisione di dati e processi tra amministrazioni, con l’obiettivo di attuare il principio \emph{once-only} (la PA non dovrebbe richiedere a cittadini e imprese dati che già possiede) \cite{piano-triennale-2024-2026}. La piattaforma è descritta come strumento che gestisce:
	\begin{itemize}[leftmargin=*]
		\item autenticazione e autorizzazione;
		\item raccolta e conservazione delle informazioni relative ad accessi e transazioni;
		\item regole condivise per semplificare gli accordi di interoperabilità, riducendo oneri e procedure.
	\end{itemize}
	
	La PDND consente alle amministrazioni di pubblicare \emph{e-service}, cioè servizi digitali erogati tramite \textbf{API} (Application Programming Interface) REST, o SOAP per retrocompatibilità, con attributi minimi necessari alla fruizione; tali API vengono registrate nel catalogo pubblico degli e-service \cite{piano-triennale-2024-2026}. La presenza di un catalogo e di regole comuni è uno degli elementi più chiari della centralizzazione di governance: non è più necessario (almeno in teoria) negoziare ogni integrazione in modo completamente bilaterale.
	
	\subsection{Evoluzioni previste per PDND: bulk, asincrono, caching, delega}
	I documenti riportano anche l’elenco di evoluzioni attese per la PDND nel triennio:
	\begin{enumerate}[leftmargin=*]
		\item condivisione di dati di grandi dimensioni (bulk) e politiche data-driven;
		\item integrazione con dati di soggetti privati non amministrativi;
		\item notifiche asincrone di variazioni su dati precedentemente fruiti e caching locale intelligente;
		\item modelli di erogazione inversa (ricevere dati da altri soggetti);
		\item scambio sincrono e asincrono, incluso trasferimento di grosse moli o pacchetti a elevati tempi di elaborazione;
		\item delega ad un aderente per operazioni sul catalogo e gestione delle richieste, inclusa analisi dei rischi;
		\item pubblicazione open data via API con catalogazione secondo norme pertinenti.
	\end{enumerate}
	Tale elenco è importante perché mostra che l’interoperabilità non è considerata come una singola API ``di consultazione'', ma come un insieme di modalità di scambio dati con complessità crescente, includendo aspetti operativi (delega, analisi rischi, caching, bulk) \cite{piano-triennale-2024-2026}.
	
	\subsection{Piattaforme nazionali e basi dati di interesse nazionale}
	Il Piano Triennale descrive anche un ecosistema di piattaforme nazionali che erogano servizi a cittadini, imprese e altre PA: ad esempio pagoPA, App IO, SEND (Servizio Notifiche Digitali), SPID (Sistema Pubblico di Identità Digitale), CIE (Carta di Identità Elettronica), NoiPA, Fascicolo Sanitario Elettronico (FSE) 2.0, SUAP/SUE e basi dati di interesse nazionale come ANPR (Anagrafe Nazionale della Popolazione Residente) \cite{piano-triennale-2024-2026}. Questi componenti hanno un obiettivo comune: migliorare servizi esistenti e uniformare esperienze e processi.
	
	In questa sede interessa soprattutto un effetto sistemico: piattaforme nazionali e basi dati centrali spostano alcune funzioni ``core'' dal livello dell’ente al livello nazionale, rendendo più possibile l’uniformità ma anche aumentando l’importanza di alcuni nodi del sistema-paese.
	
	\section{Infrastrutture e cloud: strategia Cloud Italia e PSN}
	La sezione infrastrutture del Piano Triennale inquadra la strategia \textbf{Cloud Italia} (pubblicata nel 2021 da Dipartimento per la Trasformazione Digitale e Agenzia per la Cybersicurezza Nazionale) come un percorso attuativo che risponde a tre sfide: autonomia tecnologica, controllo sui dati e resilienza dei servizi \cite{piano-triennale-2024-2026}. Il principio \emph{cloud-first} viene indicato come criterio prioritario: per nuovi progetti o nuovi servizi si deve valutare in via prioritaria l’adozione del cloud, motivando eventuali esiti negativi.
	
	La stessa sezione sottolinea che non si tratta di una sola operazione tecnologica: la migrazione è anche occasione per ridurre debito tecnologico, mitigare lock-in, ridurre costi di manutenzione di data center obsoleti e incrementare la postura di sicurezza \cite{piano-triennale-2024-2026}.
	
	\subsection{PSN: Polo Strategico Nazionale}
	Il \textbf{PSN} (Polo Strategico Nazionale) viene descritto nel Piano Triennale come infrastruttura ad alta affidabilità localizzata sul territorio nazionale, promossa per la razionalizzazione e il consolidamento dei Centri di Elaborazione Dati (CED) della PA, garantendo qualità, sicurezza, scalabilità, sostenibilità economica e continuità operativa \cite{piano-triennale-2024-2026}. Il Piano richiama anche la metodologia di classificazione di dati e servizi e la qualificazione dei fornitori di cloud pubblico, così da indirizzare la migrazione verso PSN o verso cloud qualificati coerenti con requisiti.
	
	Qui emerge un punto centrale: la transizione non significa ``tutto su PSN''. È previsto un insieme di soluzioni (PSN, infrastrutture della PA adeguate, cloud qualificati), con criteri di classificazione e qualificazione. È una centralizzazione di indirizzo e di regole, più che una centralizzazione fisica assoluta.
	
	\subsection{SPC: Sistema Pubblico di Connettività}
	Il Piano richiama il \textbf{SPC} (Sistema Pubblico di Connettività) come rete e insieme di servizi di connettività che consente l’interscambio riservato di informazioni, l’interoperabilità tra reti delle amministrazioni e l’interconnessione a Internet \cite{piano-triennale-2024-2026}. Tuttavia, lo stesso testo osserva che SPC deve trovare evoluzione nella nuova logica cloud: viene esplicitata la natura del cloud come decentrata, policentrica e federata, in grado di rendere possibile il disegno originario dello SPC salvaguardando autonomia degli enti e neutralità tecnologica \cite{piano-triennale-2024-2026}.
	
	Questo passaggio è rilevante perché mostra esplicitamente la coesistenza di due esigenze: ridurre frammentazione e aumentare coordinamento (spinta centralizzante), ma farlo attraverso un paradigma tecnologico che di per sé è distribuito e può essere federato.
	
	\section{Il documento PSN 2025: servizi, sovranità, resilienza e modelli di erogazione}
	La \emph{Descrizione nuovi servizi 2025} del PSN \cite{psn-servizi-2025} fornisce un livello di dettaglio tecnico-operativo che completa la visione del Piano Triennale. In particolare, esplicita che l’offerta PSN include più aree e modelli: \emph{PSN Managed}, \emph{Secure Public Cloud} (SPC) e \emph{Industry Standard}. La presenza di questi diversi modelli rende il quadro più chiaramente ibrido: non esiste un’unica soluzione, ma un catalogo che integra tecnologie e provider diversi, aggiungendo presidi di governance e sicurezza.
	
	\subsection{PSN Managed (Oracle Alloy): resilienza intra-region e DR inter-region}
	Nel documento, una parte rilevante descrive l’estensione dei servizi PSN Managed basati su tecnologia Oracle Alloy. Il testo insiste su aspetti come residenza dei dati in Italia, isolamento fisico/logico dei workload, controlli di sicurezza avanzati, gestione delle chiavi e integrazione con servizi PSN (identità, logging, backup, sicurezza) \cite{psn-servizi-2025}.
	
	Dal punto di vista della continuità operativa, sono richiamati concetti come:
	\begin{itemize}[leftmargin=*]
		\item fault domain (con ridondanza e anti-affinity),
		\item replica storage,
		\item replica database (Oracle RAC),
		\item assenza di downtime programmato per maintenance,
		\item disponibilità di più region (Nord e Sud) e possibilità di disaster recovery tra region con RPO definiti in convenzione.
	\end{itemize}
	Questi elementi descrivono un’infrastruttura distribuita e resiliente, che però viene resa ``sovrana'' e governata tramite controlli e gestione operativa in carico al PSN \cite{psn-servizi-2025}.
	
	\subsection{Secure Public Cloud: hyperscaler con controlli di governance e sovranità}
	Il documento descrive il \textbf{Secure Public Cloud} come accesso a region pubbliche di hyperscaler selezionati, con l’aggiunta di elementi di sicurezza e governance erogati dai data center PSN. Viene presentata un’architettura a due componenti: la parte public cloud (necessariamente su territorio nazionale) e la componente security \& governance (gestione chiavi, backup, policy, controllo traffico, ecc.) \cite{psn-servizi-2025}.
	
	Tra gli elementi chiave citati:
	\begin{itemize}[leftmargin=*]
		\item gestione chiavi di crittografia esterna al controllo del cloud provider;
		\item modello di governance e ambienti segregati standard;
		\item controlli su traffico (hub \& spoke) e monitoraggio;
		\item backup conservati nel perimetro PSN;
		\item scenari di confidential computing (in particolare nella variante Confidential Azure) con meccanismi aggiuntivi di trasparenza e controllo.
	\end{itemize}
	
	Anche qui, la logica è coerente con il carattere ibrido: si sfrutta l'infrastruttura distribuita degli hyperscaler, ma si tenta di riportare controllo e auditabilità entro un perimetro nazionale e contrattuale \cite{psn-servizi-2025}.
	
	\subsection{Osservabilità e logging: da requisito tecnico a requisito di governance}
	Il documento PSN include riferimenti espliciti a logging, logging analytics, monitoring, notification e strumenti di observability, oltre a servizi di security monitoring e assessment \cite{psn-servizi-2025}. Questo è un punto importante perché la visibilità operativa diventa condizione per la governance: senza log e audit coerenti, non è possibile né sicurezza né accountability.
	
	\section{Sicurezza e gestione del rischio nella visione del Piano Triennale}
	Il Piano Triennale dedica un capitolo alla sicurezza informatica, inquadrandola come elemento necessario per garantire la sicurezza del Paese e la continuità dei servizi pubblici \cite{piano-triennale-2024-2026}. Viene richiamata la riforma dell’architettura nazionale cyber con la nascita dell’\textbf{ACN} (Agenzia per la Cybersicurezza Nazionale), e viene indicata la necessità di:
	\begin{itemize}[leftmargin=*]
		\item modelli di gestione centralizzati della cybersicurezza;
		\item processi di gestione e mitigazione del rischio cyber, inclusa gestione delle terze parti;
		\item promozione della cultura cyber;
		\item servizi e piattaforme (es. CERT) a supporto della PA.
	\end{itemize}
	Questi elementi, letti insieme a PSN e PDND, rafforzano l’idea che la centralizzazione non sia solo infrastrutturale, ma anche di \emph{processi} e di \emph{metodi}.
	
	\section{Caso di studio operativo: estratti AGENAS/Sogei (processi, DR, SOC, IAM)}
	Negli estratti del protocollo AGENAS relativi ai servizi Sogei, compaiono elementi tipici di una gestione ICT industrializzata: processi certificati (ISO 9001, ISO/IEC 27001), ALM (Application Lifecycle Management), metodologie Agile/DevOps, misurazione dell’effort, customer care strutturato, e servizi di continuità operativa e disaster recovery con test periodici \cite{pnrr-sigeco-dtd-v5}.
	
	Sono presenti dettagli operativi particolarmente rilevanti:
	\begin{itemize}[leftmargin=*]
		\item \textbf{Disaster recovery}: replica asincrona, verifiche semestrali e prove annuali con utenti selezionati;
		\item \textbf{SOC (Security Operation Center)}: centralizzazione log via SIEM, monitoraggio real-time, incident response, patch management, reportistica;
		\item \textbf{IAM (Identity and Access Management)}: gestione identità/autorizzazioni e componenti di certificazione.
	\end{itemize}
	
	Questo caso è utile per comprendere un aspetto spesso implicito nei documenti di policy: una piattaforma nazionale o un cloud ``sicuro'' non è soltanto hardware e software, ma richiede un apparato di operation, strumenti e processi, e soprattutto la capacità di eseguire test e gestire incidenti in modo ripetibile. È anche il punto in cui la centralizzazione diventa concreta: quando logging, incident response e DR sono gestiti con procedure comuni, si riduce l’eterogeneità tra enti, ma si crea anche una dipendenza dall’effettiva qualità di tali presidi.
	
	\section{In che senso il modello tende alla centralizzazione (pur restando ibrido)?}
	
	\subsection{Centralizzazione come governance e piattaforme comuni}
	Gli elementi più chiaramente centralizzanti sono:
	\begin{itemize}[leftmargin=*]
		\item la PDND come layer nazionale di regole e catalogo per l’interoperabilità (accordi, autorizzazioni, tracciamento);
		\item la diffusione e il consolidamento di piattaforme nazionali per identità, pagamenti, notifiche e basi dati di interesse nazionale;
		\item il PSN come infrastruttura di riferimento per la migrazione di dati/servizi e come volano per la razionalizzazione dei CED.
	\end{itemize}
	
	In questo senso, la centralizzazione è una risposta alla frammentazione: ridurre il numero di ``modi diversi'' in cui si fanno le stesse cose, introducendo un insieme comune di regole, servizi e presidi.
	
	\subsection{Ibridazione: distribuzione tecnica e pluralità di modelli}
	Allo stesso tempo, il modello non elimina la distribuzione tecnica, anzi la incorpora:
	\begin{itemize}[leftmargin=*]
		\item cloud e multi-region con replica e DR;
		\item coesistenza di PSN Managed e Secure Public Cloud con hyperscaler;
		\item possibilità di infrastrutture cloud federate della PA e concetti di edge/infrastrutture di prossimità (citati nel Piano Triennale) \cite{piano-triennale-2024-2026}.
	\end{itemize}
	
	Ne risulta un assetto che può essere definito ``ibrido'' non come compromesso generico, ma perché la governance e alcuni servizi diventano più centrali, mentre l’esecuzione tecnica rimane distribuita per ragioni di resilienza, scalabilità e architettura cloud.
	
	\section{Fase critica: vulnerabilità, rischi e trade-off del modello proposto}
	La parte descrittiva mostra obiettivi coerenti e motivazioni solide (riduzione frammentazione, aumento resilienza e sicurezza). Tuttavia, un modello che accentra regole, piattaforme e (in parte) infrastrutture introduce anche criticità.
	
	\subsection{Concentrazione del rischio: ``high-value target'' e blast radius}
	Quando dati e servizi diventano accessibili tramite pochi snodi (piattaforme comuni, identità, interoperabilità), l’impatto di un incidente può crescere. Non è necessario che tutto stia fisicamente in un unico luogo: basta che esistano punti di controllo logico (ad esempio la gestione delle identità, i sistemi di autorizzazione, i cataloghi, i sistemi di logging e i meccanismi di trust) affinché un attacco ben riuscito possa avere un \emph{blast radius} maggiore.
	
	Questo porta a una prima vulnerabilità strutturale: l’incentivo ad attacchi mirati aumenta. Un attaccante razionale tende a scegliere l’obiettivo che massimizza impatto e ritorno. Se l’ecosistema diventa più uniforme e più interconnesso, i nodi critici diventano più evidenti e appetibili.
	
	\subsection{Fattore umano: credenziali, errori, procedure e social engineering}
	Questa architettura fa pensare che il fattore umano rimane l’elemento più vulnerabile. Anche con infrastruttura robusta e cifratura, un attacco che compromette credenziali privilegiate, processi di change management o account di supporto può aggirare molti controlli.
	
	Il rischio cresce quando:
	\begin{itemize}[leftmargin=*]
		\item aumenta il numero di attori coinvolti (enti, fornitori, subfornitori, operatori);
		\item aumentano i privilegi necessari per operare (gestione tenant, gestione chiavi, emergenze);
		\item non esistono controlli stringenti su accessi amministrativi, segregazione dei compiti e audit.
	\end{itemize}
	
	In pratica, non basta definire policy: serve una disciplina continua nell’esecuzione, soprattutto su identità e accessi (IAM), gestione privilegi (PAM), e gestione delle emergenze (break-glass accounts).
	
	\subsection{Dipendenza operativa: qualità reale di SOC, logging, DR e incident response}
	I documenti evidenziano strumenti e servizi (logging, monitoring, DR, SOC), ma il rischio non è soltanto l’assenza di questi strumenti: è anche la loro efficacia reale. Un SOC esiste davvero quando:
	\begin{itemize}[leftmargin=*]
		\item riceve log completi e corretti (non solo infrastruttura, anche applicazioni);
		\item correla e genera alert utili (riduzione falsi positivi / falsi negativi);
		\item ha processi di incident response e canali decisionali rapidi;
		\item esegue esercitazioni e post-mortem che producono miglioramenti.
	\end{itemize}
	
	Gli estratti AGENAS/Sogei mostrano un esempio di impostazione matura (test periodici, SIEM, procedure). Tuttavia, il punto critico è sistemico: se molte amministrazioni dipendono da una catena operativa comune, la qualità di quella catena diventa determinante.
	
	\subsection{Gestione delle terze parti e supply chain}
	Il modello proposto include hyperscaler e componenti di mercato; inoltre, molte PA continueranno a usare fornitori esterni per sviluppo, migrazione e manutenzione. Questo introduce un rischio di supply chain: vulnerabilità in software di terze parti, catene CI/CD, librerie, tool di gestione, e accessi di supporto.
	
	I documenti richiamano la necessità di processi di gestione del rischio anche rispetto alle terze parti \cite{piano-triennale-2024-2026}. La criticità qui è che la gestione delle terze parti non è un adempimento burocratico: richiede inventario aggiornato, controllo degli accessi, segmentazione, audit, e capacità di revoca rapida.
	
	\subsection{Interoperabilità: la fragilità dei contratti e della semantica}
	La PDND introduce un catalogo e regole comuni. Tuttavia, l’interoperabilità può fallire anche se ``l’API esiste'', quando:
	\begin{itemize}[leftmargin=*]
		\item la semantica dei dati è ambigua o incoerente tra enti;
		\item mancano regole chiare di versioning e deprecazione;
		\item il dato fonte cambia senza un meccanismo affidabile di notifica/aggiornamento;
		\item i fruitori implementano cache o repliche senza governance.
	\end{itemize}
	
	Gli appunti citano evoluzioni (notifiche asincrone e caching intelligente) \cite{piano-triennale-2024-2026} proprio perché il problema è noto: l’interoperabilità a scala nazionale non è solo ``mettere online una query'', ma garantire continuità, aggiornamento e comprensione condivisa del dato.
	
	\subsection{Resilienza e disponibilità: la ridondanza non basta senza prove}
	La resilienza promessa dal cloud e dal PSN dipende da come vengono configurati i sistemi applicativi. Il documento PSN parla di fault domain e region, e cita DR inter-region come possibilità \cite{psn-servizi-2025}. Tuttavia:
	\begin{itemize}[leftmargin=*]
		\item un’applicazione può essere ospitata in cloud ma restare fragile (dipendenze esterne, configurazioni errate, single points of failure logici);
		\item un piano di DR senza prove periodiche può fallire nel momento del bisogno;
		\item la gestione dei backup può essere inefficace se non verificata (restore test, integrity, tempi).
	\end{itemize}
	
	Gli estratti AGENAS/Sogei sottolineano test semestrali e annuali: questo è un indicatore concreto di maturità. La criticità è che tale maturità deve essere generalizzata, altrimenti la resilienza rimane teorica o disomogenea.
	
	\subsection{Lock-in: non solo tecnologico, ma anche di processo e competenze}
	Il Piano Triennale cita la mitigazione del lock-in come obiettivo della strategia cloud \cite{piano-triennale-2024-2026}. È un obiettivo realistico solo se si riconosce che il lock-in ha almeno tre componenti:
	\begin{itemize}[leftmargin=*]
		\item \textbf{tecnologica}: uso di servizi proprietari difficili da migrare;
		\item \textbf{operativa}: strumenti e runbook costruiti intorno a una piattaforma;
		\item \textbf{organizzativa}: competenze concentrate su uno stack, difficili da sostituire.
	\end{itemize}
	
	Il rischio è che la PA migliori la propria resilienza e sicurezza rispetto a data center obsoleti, ma si trovi vincolata a scelte difficilmente reversibili in futuro (non necessariamente per malafede, ma per inerzia tecnica e organizzativa).
	
	\section{Condizioni di successo}
	Dai documenti emerge un disegno ambizioso e, per molti aspetti, coerente: ridurre frammentazione, aumentare resilienza, abilitare interoperabilità e ``once-only'', e costruire una sicurezza più strutturata. Tuttavia, la trasformazione produce anche nuove forme di rischio legate alla concentrazione di snodi critici, alla dipendenza operativa e alla supply chain.
	
	Perché i benefici attesi siano realmente ottenuti, alcune condizioni appaiono essenziali:
	\begin{itemize}[leftmargin=*]
		\item \textbf{Disciplina sugli accessi}: identità e privilegi devono essere governati con rigore, audit e segregazione dei compiti.
		\item \textbf{Osservabilità e audit end-to-end}: logging e monitoring non solo infrastrutturali ma anche applicativi, con procedure di risposta agli incidenti.
		\item \textbf{DR verificato}: backup e disaster recovery devono essere testati periodicamente e non solo dichiarati.
		\item \textbf{Interoperabilità come contratto}: API e e-service devono avere versioning, semantica chiara e gestione dell’evoluzione.
		\item \textbf{Gestione fornitori}: controllo delle terze parti come parte integrante della postura di sicurezza.
	\end{itemize}
	
	In conclusione, l’assetto che emerge dai documenti può essere descritto come un passaggio da una frammentazione storica verso un modello più coordinato e standardizzato, con una chiara spinta verso la centralizzazione della governance e delle piattaforme, ma con una realizzazione necessariamente distribuita e plurale. Proprio questa natura ibrida rende il progetto promettente, ma anche complesso: i rischi non spariscono, cambiano forma, e richiedono capacità operative e organizzative all’altezza dell’ambizione.
	
	%\printbibliography
	
\end{document}